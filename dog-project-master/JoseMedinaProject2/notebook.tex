
% Default to the notebook output style

    


% Inherit from the specified cell style.




    
\documentclass[11pt]{article}

    
    
    \usepackage[T1]{fontenc}
    % Nicer default font (+ math font) than Computer Modern for most use cases
    \usepackage{mathpazo}

    % Basic figure setup, for now with no caption control since it's done
    % automatically by Pandoc (which extracts ![](path) syntax from Markdown).
    \usepackage{graphicx}
    % We will generate all images so they have a width \maxwidth. This means
    % that they will get their normal width if they fit onto the page, but
    % are scaled down if they would overflow the margins.
    \makeatletter
    \def\maxwidth{\ifdim\Gin@nat@width>\linewidth\linewidth
    \else\Gin@nat@width\fi}
    \makeatother
    \let\Oldincludegraphics\includegraphics
    % Set max figure width to be 80% of text width, for now hardcoded.
    \renewcommand{\includegraphics}[1]{\Oldincludegraphics[width=.8\maxwidth]{#1}}
    % Ensure that by default, figures have no caption (until we provide a
    % proper Figure object with a Caption API and a way to capture that
    % in the conversion process - todo).
    \usepackage{caption}
    \DeclareCaptionLabelFormat{nolabel}{}
    \captionsetup{labelformat=nolabel}

    \usepackage{adjustbox} % Used to constrain images to a maximum size 
    \usepackage{xcolor} % Allow colors to be defined
    \usepackage{enumerate} % Needed for markdown enumerations to work
    \usepackage{geometry} % Used to adjust the document margins
    \usepackage{amsmath} % Equations
    \usepackage{amssymb} % Equations
    \usepackage{textcomp} % defines textquotesingle
    % Hack from http://tex.stackexchange.com/a/47451/13684:
    \AtBeginDocument{%
        \def\PYZsq{\textquotesingle}% Upright quotes in Pygmentized code
    }
    \usepackage{upquote} % Upright quotes for verbatim code
    \usepackage{eurosym} % defines \euro
    \usepackage[mathletters]{ucs} % Extended unicode (utf-8) support
    \usepackage[utf8x]{inputenc} % Allow utf-8 characters in the tex document
    \usepackage{fancyvrb} % verbatim replacement that allows latex
    \usepackage{grffile} % extends the file name processing of package graphics 
                         % to support a larger range 
    % The hyperref package gives us a pdf with properly built
    % internal navigation ('pdf bookmarks' for the table of contents,
    % internal cross-reference links, web links for URLs, etc.)
    \usepackage{hyperref}
    \usepackage{longtable} % longtable support required by pandoc >1.10
    \usepackage{booktabs}  % table support for pandoc > 1.12.2
    \usepackage[inline]{enumitem} % IRkernel/repr support (it uses the enumerate* environment)
    \usepackage[normalem]{ulem} % ulem is needed to support strikethroughs (\sout)
                                % normalem makes italics be italics, not underlines
    

    
    
    % Colors for the hyperref package
    \definecolor{urlcolor}{rgb}{0,.145,.698}
    \definecolor{linkcolor}{rgb}{.71,0.21,0.01}
    \definecolor{citecolor}{rgb}{.12,.54,.11}

    % ANSI colors
    \definecolor{ansi-black}{HTML}{3E424D}
    \definecolor{ansi-black-intense}{HTML}{282C36}
    \definecolor{ansi-red}{HTML}{E75C58}
    \definecolor{ansi-red-intense}{HTML}{B22B31}
    \definecolor{ansi-green}{HTML}{00A250}
    \definecolor{ansi-green-intense}{HTML}{007427}
    \definecolor{ansi-yellow}{HTML}{DDB62B}
    \definecolor{ansi-yellow-intense}{HTML}{B27D12}
    \definecolor{ansi-blue}{HTML}{208FFB}
    \definecolor{ansi-blue-intense}{HTML}{0065CA}
    \definecolor{ansi-magenta}{HTML}{D160C4}
    \definecolor{ansi-magenta-intense}{HTML}{A03196}
    \definecolor{ansi-cyan}{HTML}{60C6C8}
    \definecolor{ansi-cyan-intense}{HTML}{258F8F}
    \definecolor{ansi-white}{HTML}{C5C1B4}
    \definecolor{ansi-white-intense}{HTML}{A1A6B2}

    % commands and environments needed by pandoc snippets
    % extracted from the output of `pandoc -s`
    \providecommand{\tightlist}{%
      \setlength{\itemsep}{0pt}\setlength{\parskip}{0pt}}
    \DefineVerbatimEnvironment{Highlighting}{Verbatim}{commandchars=\\\{\}}
    % Add ',fontsize=\small' for more characters per line
    \newenvironment{Shaded}{}{}
    \newcommand{\KeywordTok}[1]{\textcolor[rgb]{0.00,0.44,0.13}{\textbf{{#1}}}}
    \newcommand{\DataTypeTok}[1]{\textcolor[rgb]{0.56,0.13,0.00}{{#1}}}
    \newcommand{\DecValTok}[1]{\textcolor[rgb]{0.25,0.63,0.44}{{#1}}}
    \newcommand{\BaseNTok}[1]{\textcolor[rgb]{0.25,0.63,0.44}{{#1}}}
    \newcommand{\FloatTok}[1]{\textcolor[rgb]{0.25,0.63,0.44}{{#1}}}
    \newcommand{\CharTok}[1]{\textcolor[rgb]{0.25,0.44,0.63}{{#1}}}
    \newcommand{\StringTok}[1]{\textcolor[rgb]{0.25,0.44,0.63}{{#1}}}
    \newcommand{\CommentTok}[1]{\textcolor[rgb]{0.38,0.63,0.69}{\textit{{#1}}}}
    \newcommand{\OtherTok}[1]{\textcolor[rgb]{0.00,0.44,0.13}{{#1}}}
    \newcommand{\AlertTok}[1]{\textcolor[rgb]{1.00,0.00,0.00}{\textbf{{#1}}}}
    \newcommand{\FunctionTok}[1]{\textcolor[rgb]{0.02,0.16,0.49}{{#1}}}
    \newcommand{\RegionMarkerTok}[1]{{#1}}
    \newcommand{\ErrorTok}[1]{\textcolor[rgb]{1.00,0.00,0.00}{\textbf{{#1}}}}
    \newcommand{\NormalTok}[1]{{#1}}
    
    % Additional commands for more recent versions of Pandoc
    \newcommand{\ConstantTok}[1]{\textcolor[rgb]{0.53,0.00,0.00}{{#1}}}
    \newcommand{\SpecialCharTok}[1]{\textcolor[rgb]{0.25,0.44,0.63}{{#1}}}
    \newcommand{\VerbatimStringTok}[1]{\textcolor[rgb]{0.25,0.44,0.63}{{#1}}}
    \newcommand{\SpecialStringTok}[1]{\textcolor[rgb]{0.73,0.40,0.53}{{#1}}}
    \newcommand{\ImportTok}[1]{{#1}}
    \newcommand{\DocumentationTok}[1]{\textcolor[rgb]{0.73,0.13,0.13}{\textit{{#1}}}}
    \newcommand{\AnnotationTok}[1]{\textcolor[rgb]{0.38,0.63,0.69}{\textbf{\textit{{#1}}}}}
    \newcommand{\CommentVarTok}[1]{\textcolor[rgb]{0.38,0.63,0.69}{\textbf{\textit{{#1}}}}}
    \newcommand{\VariableTok}[1]{\textcolor[rgb]{0.10,0.09,0.49}{{#1}}}
    \newcommand{\ControlFlowTok}[1]{\textcolor[rgb]{0.00,0.44,0.13}{\textbf{{#1}}}}
    \newcommand{\OperatorTok}[1]{\textcolor[rgb]{0.40,0.40,0.40}{{#1}}}
    \newcommand{\BuiltInTok}[1]{{#1}}
    \newcommand{\ExtensionTok}[1]{{#1}}
    \newcommand{\PreprocessorTok}[1]{\textcolor[rgb]{0.74,0.48,0.00}{{#1}}}
    \newcommand{\AttributeTok}[1]{\textcolor[rgb]{0.49,0.56,0.16}{{#1}}}
    \newcommand{\InformationTok}[1]{\textcolor[rgb]{0.38,0.63,0.69}{\textbf{\textit{{#1}}}}}
    \newcommand{\WarningTok}[1]{\textcolor[rgb]{0.38,0.63,0.69}{\textbf{\textit{{#1}}}}}
    
    
    % Define a nice break command that doesn't care if a line doesn't already
    % exist.
    \def\br{\hspace*{\fill} \\* }
    % Math Jax compatability definitions
    \def\gt{>}
    \def\lt{<}
    % Document parameters
    \title{dog\_app}
    
    
    

    % Pygments definitions
    
\makeatletter
\def\PY@reset{\let\PY@it=\relax \let\PY@bf=\relax%
    \let\PY@ul=\relax \let\PY@tc=\relax%
    \let\PY@bc=\relax \let\PY@ff=\relax}
\def\PY@tok#1{\csname PY@tok@#1\endcsname}
\def\PY@toks#1+{\ifx\relax#1\empty\else%
    \PY@tok{#1}\expandafter\PY@toks\fi}
\def\PY@do#1{\PY@bc{\PY@tc{\PY@ul{%
    \PY@it{\PY@bf{\PY@ff{#1}}}}}}}
\def\PY#1#2{\PY@reset\PY@toks#1+\relax+\PY@do{#2}}

\expandafter\def\csname PY@tok@w\endcsname{\def\PY@tc##1{\textcolor[rgb]{0.73,0.73,0.73}{##1}}}
\expandafter\def\csname PY@tok@c\endcsname{\let\PY@it=\textit\def\PY@tc##1{\textcolor[rgb]{0.25,0.50,0.50}{##1}}}
\expandafter\def\csname PY@tok@cp\endcsname{\def\PY@tc##1{\textcolor[rgb]{0.74,0.48,0.00}{##1}}}
\expandafter\def\csname PY@tok@k\endcsname{\let\PY@bf=\textbf\def\PY@tc##1{\textcolor[rgb]{0.00,0.50,0.00}{##1}}}
\expandafter\def\csname PY@tok@kp\endcsname{\def\PY@tc##1{\textcolor[rgb]{0.00,0.50,0.00}{##1}}}
\expandafter\def\csname PY@tok@kt\endcsname{\def\PY@tc##1{\textcolor[rgb]{0.69,0.00,0.25}{##1}}}
\expandafter\def\csname PY@tok@o\endcsname{\def\PY@tc##1{\textcolor[rgb]{0.40,0.40,0.40}{##1}}}
\expandafter\def\csname PY@tok@ow\endcsname{\let\PY@bf=\textbf\def\PY@tc##1{\textcolor[rgb]{0.67,0.13,1.00}{##1}}}
\expandafter\def\csname PY@tok@nb\endcsname{\def\PY@tc##1{\textcolor[rgb]{0.00,0.50,0.00}{##1}}}
\expandafter\def\csname PY@tok@nf\endcsname{\def\PY@tc##1{\textcolor[rgb]{0.00,0.00,1.00}{##1}}}
\expandafter\def\csname PY@tok@nc\endcsname{\let\PY@bf=\textbf\def\PY@tc##1{\textcolor[rgb]{0.00,0.00,1.00}{##1}}}
\expandafter\def\csname PY@tok@nn\endcsname{\let\PY@bf=\textbf\def\PY@tc##1{\textcolor[rgb]{0.00,0.00,1.00}{##1}}}
\expandafter\def\csname PY@tok@ne\endcsname{\let\PY@bf=\textbf\def\PY@tc##1{\textcolor[rgb]{0.82,0.25,0.23}{##1}}}
\expandafter\def\csname PY@tok@nv\endcsname{\def\PY@tc##1{\textcolor[rgb]{0.10,0.09,0.49}{##1}}}
\expandafter\def\csname PY@tok@no\endcsname{\def\PY@tc##1{\textcolor[rgb]{0.53,0.00,0.00}{##1}}}
\expandafter\def\csname PY@tok@nl\endcsname{\def\PY@tc##1{\textcolor[rgb]{0.63,0.63,0.00}{##1}}}
\expandafter\def\csname PY@tok@ni\endcsname{\let\PY@bf=\textbf\def\PY@tc##1{\textcolor[rgb]{0.60,0.60,0.60}{##1}}}
\expandafter\def\csname PY@tok@na\endcsname{\def\PY@tc##1{\textcolor[rgb]{0.49,0.56,0.16}{##1}}}
\expandafter\def\csname PY@tok@nt\endcsname{\let\PY@bf=\textbf\def\PY@tc##1{\textcolor[rgb]{0.00,0.50,0.00}{##1}}}
\expandafter\def\csname PY@tok@nd\endcsname{\def\PY@tc##1{\textcolor[rgb]{0.67,0.13,1.00}{##1}}}
\expandafter\def\csname PY@tok@s\endcsname{\def\PY@tc##1{\textcolor[rgb]{0.73,0.13,0.13}{##1}}}
\expandafter\def\csname PY@tok@sd\endcsname{\let\PY@it=\textit\def\PY@tc##1{\textcolor[rgb]{0.73,0.13,0.13}{##1}}}
\expandafter\def\csname PY@tok@si\endcsname{\let\PY@bf=\textbf\def\PY@tc##1{\textcolor[rgb]{0.73,0.40,0.53}{##1}}}
\expandafter\def\csname PY@tok@se\endcsname{\let\PY@bf=\textbf\def\PY@tc##1{\textcolor[rgb]{0.73,0.40,0.13}{##1}}}
\expandafter\def\csname PY@tok@sr\endcsname{\def\PY@tc##1{\textcolor[rgb]{0.73,0.40,0.53}{##1}}}
\expandafter\def\csname PY@tok@ss\endcsname{\def\PY@tc##1{\textcolor[rgb]{0.10,0.09,0.49}{##1}}}
\expandafter\def\csname PY@tok@sx\endcsname{\def\PY@tc##1{\textcolor[rgb]{0.00,0.50,0.00}{##1}}}
\expandafter\def\csname PY@tok@m\endcsname{\def\PY@tc##1{\textcolor[rgb]{0.40,0.40,0.40}{##1}}}
\expandafter\def\csname PY@tok@gh\endcsname{\let\PY@bf=\textbf\def\PY@tc##1{\textcolor[rgb]{0.00,0.00,0.50}{##1}}}
\expandafter\def\csname PY@tok@gu\endcsname{\let\PY@bf=\textbf\def\PY@tc##1{\textcolor[rgb]{0.50,0.00,0.50}{##1}}}
\expandafter\def\csname PY@tok@gd\endcsname{\def\PY@tc##1{\textcolor[rgb]{0.63,0.00,0.00}{##1}}}
\expandafter\def\csname PY@tok@gi\endcsname{\def\PY@tc##1{\textcolor[rgb]{0.00,0.63,0.00}{##1}}}
\expandafter\def\csname PY@tok@gr\endcsname{\def\PY@tc##1{\textcolor[rgb]{1.00,0.00,0.00}{##1}}}
\expandafter\def\csname PY@tok@ge\endcsname{\let\PY@it=\textit}
\expandafter\def\csname PY@tok@gs\endcsname{\let\PY@bf=\textbf}
\expandafter\def\csname PY@tok@gp\endcsname{\let\PY@bf=\textbf\def\PY@tc##1{\textcolor[rgb]{0.00,0.00,0.50}{##1}}}
\expandafter\def\csname PY@tok@go\endcsname{\def\PY@tc##1{\textcolor[rgb]{0.53,0.53,0.53}{##1}}}
\expandafter\def\csname PY@tok@gt\endcsname{\def\PY@tc##1{\textcolor[rgb]{0.00,0.27,0.87}{##1}}}
\expandafter\def\csname PY@tok@err\endcsname{\def\PY@bc##1{\setlength{\fboxsep}{0pt}\fcolorbox[rgb]{1.00,0.00,0.00}{1,1,1}{\strut ##1}}}
\expandafter\def\csname PY@tok@kc\endcsname{\let\PY@bf=\textbf\def\PY@tc##1{\textcolor[rgb]{0.00,0.50,0.00}{##1}}}
\expandafter\def\csname PY@tok@kd\endcsname{\let\PY@bf=\textbf\def\PY@tc##1{\textcolor[rgb]{0.00,0.50,0.00}{##1}}}
\expandafter\def\csname PY@tok@kn\endcsname{\let\PY@bf=\textbf\def\PY@tc##1{\textcolor[rgb]{0.00,0.50,0.00}{##1}}}
\expandafter\def\csname PY@tok@kr\endcsname{\let\PY@bf=\textbf\def\PY@tc##1{\textcolor[rgb]{0.00,0.50,0.00}{##1}}}
\expandafter\def\csname PY@tok@bp\endcsname{\def\PY@tc##1{\textcolor[rgb]{0.00,0.50,0.00}{##1}}}
\expandafter\def\csname PY@tok@fm\endcsname{\def\PY@tc##1{\textcolor[rgb]{0.00,0.00,1.00}{##1}}}
\expandafter\def\csname PY@tok@vc\endcsname{\def\PY@tc##1{\textcolor[rgb]{0.10,0.09,0.49}{##1}}}
\expandafter\def\csname PY@tok@vg\endcsname{\def\PY@tc##1{\textcolor[rgb]{0.10,0.09,0.49}{##1}}}
\expandafter\def\csname PY@tok@vi\endcsname{\def\PY@tc##1{\textcolor[rgb]{0.10,0.09,0.49}{##1}}}
\expandafter\def\csname PY@tok@vm\endcsname{\def\PY@tc##1{\textcolor[rgb]{0.10,0.09,0.49}{##1}}}
\expandafter\def\csname PY@tok@sa\endcsname{\def\PY@tc##1{\textcolor[rgb]{0.73,0.13,0.13}{##1}}}
\expandafter\def\csname PY@tok@sb\endcsname{\def\PY@tc##1{\textcolor[rgb]{0.73,0.13,0.13}{##1}}}
\expandafter\def\csname PY@tok@sc\endcsname{\def\PY@tc##1{\textcolor[rgb]{0.73,0.13,0.13}{##1}}}
\expandafter\def\csname PY@tok@dl\endcsname{\def\PY@tc##1{\textcolor[rgb]{0.73,0.13,0.13}{##1}}}
\expandafter\def\csname PY@tok@s2\endcsname{\def\PY@tc##1{\textcolor[rgb]{0.73,0.13,0.13}{##1}}}
\expandafter\def\csname PY@tok@sh\endcsname{\def\PY@tc##1{\textcolor[rgb]{0.73,0.13,0.13}{##1}}}
\expandafter\def\csname PY@tok@s1\endcsname{\def\PY@tc##1{\textcolor[rgb]{0.73,0.13,0.13}{##1}}}
\expandafter\def\csname PY@tok@mb\endcsname{\def\PY@tc##1{\textcolor[rgb]{0.40,0.40,0.40}{##1}}}
\expandafter\def\csname PY@tok@mf\endcsname{\def\PY@tc##1{\textcolor[rgb]{0.40,0.40,0.40}{##1}}}
\expandafter\def\csname PY@tok@mh\endcsname{\def\PY@tc##1{\textcolor[rgb]{0.40,0.40,0.40}{##1}}}
\expandafter\def\csname PY@tok@mi\endcsname{\def\PY@tc##1{\textcolor[rgb]{0.40,0.40,0.40}{##1}}}
\expandafter\def\csname PY@tok@il\endcsname{\def\PY@tc##1{\textcolor[rgb]{0.40,0.40,0.40}{##1}}}
\expandafter\def\csname PY@tok@mo\endcsname{\def\PY@tc##1{\textcolor[rgb]{0.40,0.40,0.40}{##1}}}
\expandafter\def\csname PY@tok@ch\endcsname{\let\PY@it=\textit\def\PY@tc##1{\textcolor[rgb]{0.25,0.50,0.50}{##1}}}
\expandafter\def\csname PY@tok@cm\endcsname{\let\PY@it=\textit\def\PY@tc##1{\textcolor[rgb]{0.25,0.50,0.50}{##1}}}
\expandafter\def\csname PY@tok@cpf\endcsname{\let\PY@it=\textit\def\PY@tc##1{\textcolor[rgb]{0.25,0.50,0.50}{##1}}}
\expandafter\def\csname PY@tok@c1\endcsname{\let\PY@it=\textit\def\PY@tc##1{\textcolor[rgb]{0.25,0.50,0.50}{##1}}}
\expandafter\def\csname PY@tok@cs\endcsname{\let\PY@it=\textit\def\PY@tc##1{\textcolor[rgb]{0.25,0.50,0.50}{##1}}}

\def\PYZbs{\char`\\}
\def\PYZus{\char`\_}
\def\PYZob{\char`\{}
\def\PYZcb{\char`\}}
\def\PYZca{\char`\^}
\def\PYZam{\char`\&}
\def\PYZlt{\char`\<}
\def\PYZgt{\char`\>}
\def\PYZsh{\char`\#}
\def\PYZpc{\char`\%}
\def\PYZdl{\char`\$}
\def\PYZhy{\char`\-}
\def\PYZsq{\char`\'}
\def\PYZdq{\char`\"}
\def\PYZti{\char`\~}
% for compatibility with earlier versions
\def\PYZat{@}
\def\PYZlb{[}
\def\PYZrb{]}
\makeatother


    % Exact colors from NB
    \definecolor{incolor}{rgb}{0.0, 0.0, 0.5}
    \definecolor{outcolor}{rgb}{0.545, 0.0, 0.0}



    
    % Prevent overflowing lines due to hard-to-break entities
    \sloppy 
    % Setup hyperref package
    \hypersetup{
      breaklinks=true,  % so long urls are correctly broken across lines
      colorlinks=true,
      urlcolor=urlcolor,
      linkcolor=linkcolor,
      citecolor=citecolor,
      }
    % Slightly bigger margins than the latex defaults
    
    \geometry{verbose,tmargin=1in,bmargin=1in,lmargin=1in,rmargin=1in}
    
    

    \begin{document}
    
    
    \maketitle
    
    

    
    \section{Artificial Intelligence
Nanodegree}\label{artificial-intelligence-nanodegree}

\subsection{Convolutional Neural
Networks}\label{convolutional-neural-networks}

\subsection{Project: Write an Algorithm for a Dog Identification
App}\label{project-write-an-algorithm-for-a-dog-identification-app}

\begin{center}\rule{0.5\linewidth}{\linethickness}\end{center}

In this notebook, some template code has already been provided for you,
and you will need to implement additional functionality to successfully
complete this project. You will not need to modify the included code
beyond what is requested. Sections that begin with
\textbf{'(IMPLEMENTATION)'} in the header indicate that the following
block of code will require additional functionality which you must
provide. Instructions will be provided for each section, and the
specifics of the implementation are marked in the code block with a
'TODO' statement. Please be sure to read the instructions carefully!

\begin{quote}
\textbf{Note}: Once you have completed all of the code implementations,
you need to finalize your work by exporting the iPython Notebook as an
HTML document. Before exporting the notebook to html, all of the code
cells need to have been run so that reviewers can see the final
implementation and output. You can then export the notebook by using the
menu above and navigating to \n", "\textbf{File -\textgreater{} Download
as -\textgreater{} HTML (.html)}. Include the finished document along
with this notebook as your submission.
\end{quote}

In addition to implementing code, there will be questions that you must
answer which relate to the project and your implementation. Each section
where you will answer a question is preceded by a \textbf{'Question X'}
header. Carefully read each question and provide thorough answers in the
following text boxes that begin with \textbf{'Answer:'}. Your project
submission will be evaluated based on your answers to each of the
questions and the implementation you provide.

\begin{quote}
\textbf{Note:} Code and Markdown cells can be executed using the
\textbf{Shift + Enter} keyboard shortcut. Markdown cells can be edited
by double-clicking the cell to enter edit mode.
\end{quote}

The rubric contains \emph{optional} "Stand Out Suggestions" for
enhancing the project beyond the minimum requirements. If you decide to
pursue the "Stand Out Suggestions", you should include the code in this
IPython notebook.

 \#\# Step 0: Import Datasets

\subsubsection{Import Dog Dataset}\label{import-dog-dataset}

In the code cell below, we import a dataset of dog images. We populate a
few variables through the use of the \texttt{load\_files} function from
the scikit-learn library: - \texttt{train\_files},
\texttt{valid\_files}, \texttt{test\_files} - numpy arrays containing
file paths to images - \texttt{train\_targets}, \texttt{valid\_targets},
\texttt{test\_targets} - numpy arrays containing onehot-encoded
classification labels - \texttt{dog\_names} - list of string-valued dog
breed names for translating labels

    \begin{Verbatim}[commandchars=\\\{\}]
{\color{incolor}In [{\color{incolor}1}]:} \PY{k+kn}{from} \PY{n+nn}{sklearn}\PY{n+nn}{.}\PY{n+nn}{datasets} \PY{k}{import} \PY{n}{load\PYZus{}files}       
        \PY{k+kn}{from} \PY{n+nn}{keras}\PY{n+nn}{.}\PY{n+nn}{utils} \PY{k}{import} \PY{n}{np\PYZus{}utils}
        \PY{k+kn}{import} \PY{n+nn}{numpy} \PY{k}{as} \PY{n+nn}{np}
        \PY{k+kn}{from} \PY{n+nn}{glob} \PY{k}{import} \PY{n}{glob}
        
        \PY{c+c1}{\PYZsh{} define function to load train, test, and validation datasets}
        \PY{k}{def} \PY{n+nf}{load\PYZus{}dataset}\PY{p}{(}\PY{n}{path}\PY{p}{)}\PY{p}{:}
            \PY{n}{data} \PY{o}{=} \PY{n}{load\PYZus{}files}\PY{p}{(}\PY{n}{path}\PY{p}{)}
            \PY{n}{dog\PYZus{}files} \PY{o}{=} \PY{n}{np}\PY{o}{.}\PY{n}{array}\PY{p}{(}\PY{n}{data}\PY{p}{[}\PY{l+s+s1}{\PYZsq{}}\PY{l+s+s1}{filenames}\PY{l+s+s1}{\PYZsq{}}\PY{p}{]}\PY{p}{)}
            \PY{n}{dog\PYZus{}targets} \PY{o}{=} \PY{n}{np\PYZus{}utils}\PY{o}{.}\PY{n}{to\PYZus{}categorical}\PY{p}{(}\PY{n}{np}\PY{o}{.}\PY{n}{array}\PY{p}{(}\PY{n}{data}\PY{p}{[}\PY{l+s+s1}{\PYZsq{}}\PY{l+s+s1}{target}\PY{l+s+s1}{\PYZsq{}}\PY{p}{]}\PY{p}{)}\PY{p}{,} \PY{l+m+mi}{133}\PY{p}{)}
            \PY{k}{return} \PY{n}{dog\PYZus{}files}\PY{p}{,} \PY{n}{dog\PYZus{}targets}
        
        \PY{c+c1}{\PYZsh{} load train, test, and validation datasets}
        \PY{n}{train\PYZus{}files}\PY{p}{,} \PY{n}{train\PYZus{}targets} \PY{o}{=} \PY{n}{load\PYZus{}dataset}\PY{p}{(}\PY{l+s+s1}{\PYZsq{}}\PY{l+s+s1}{dogImages/train}\PY{l+s+s1}{\PYZsq{}}\PY{p}{)}
        \PY{n}{valid\PYZus{}files}\PY{p}{,} \PY{n}{valid\PYZus{}targets} \PY{o}{=} \PY{n}{load\PYZus{}dataset}\PY{p}{(}\PY{l+s+s1}{\PYZsq{}}\PY{l+s+s1}{dogImages/valid}\PY{l+s+s1}{\PYZsq{}}\PY{p}{)}
        \PY{n}{test\PYZus{}files}\PY{p}{,} \PY{n}{test\PYZus{}targets} \PY{o}{=} \PY{n}{load\PYZus{}dataset}\PY{p}{(}\PY{l+s+s1}{\PYZsq{}}\PY{l+s+s1}{dogImages/test}\PY{l+s+s1}{\PYZsq{}}\PY{p}{)}
        
        \PY{c+c1}{\PYZsh{} load list of dog names}
        \PY{n}{dog\PYZus{}names} \PY{o}{=} \PY{p}{[}\PY{n}{item}\PY{p}{[}\PY{l+m+mi}{20}\PY{p}{:}\PY{o}{\PYZhy{}}\PY{l+m+mi}{1}\PY{p}{]} \PY{k}{for} \PY{n}{item} \PY{o+ow}{in} \PY{n+nb}{sorted}\PY{p}{(}\PY{n}{glob}\PY{p}{(}\PY{l+s+s2}{\PYZdq{}}\PY{l+s+s2}{dogImages/train/*/}\PY{l+s+s2}{\PYZdq{}}\PY{p}{)}\PY{p}{)}\PY{p}{]}
        
        \PY{c+c1}{\PYZsh{} print statistics about the dataset}
        \PY{n+nb}{print}\PY{p}{(}\PY{l+s+s1}{\PYZsq{}}\PY{l+s+s1}{There are }\PY{l+s+si}{\PYZpc{}d}\PY{l+s+s1}{ total dog categories.}\PY{l+s+s1}{\PYZsq{}} \PY{o}{\PYZpc{}} \PY{n+nb}{len}\PY{p}{(}\PY{n}{dog\PYZus{}names}\PY{p}{)}\PY{p}{)}
        \PY{n+nb}{print}\PY{p}{(}\PY{l+s+s1}{\PYZsq{}}\PY{l+s+s1}{There are }\PY{l+s+si}{\PYZpc{}s}\PY{l+s+s1}{ total dog images.}\PY{l+s+se}{\PYZbs{}n}\PY{l+s+s1}{\PYZsq{}} \PY{o}{\PYZpc{}} \PY{n+nb}{len}\PY{p}{(}\PY{n}{np}\PY{o}{.}\PY{n}{hstack}\PY{p}{(}\PY{p}{[}\PY{n}{train\PYZus{}files}\PY{p}{,} \PY{n}{valid\PYZus{}files}\PY{p}{,} \PY{n}{test\PYZus{}files}\PY{p}{]}\PY{p}{)}\PY{p}{)}\PY{p}{)}
        \PY{n+nb}{print}\PY{p}{(}\PY{l+s+s1}{\PYZsq{}}\PY{l+s+s1}{There are }\PY{l+s+si}{\PYZpc{}d}\PY{l+s+s1}{ training dog images.}\PY{l+s+s1}{\PYZsq{}} \PY{o}{\PYZpc{}} \PY{n+nb}{len}\PY{p}{(}\PY{n}{train\PYZus{}files}\PY{p}{)}\PY{p}{)}
        \PY{n+nb}{print}\PY{p}{(}\PY{l+s+s1}{\PYZsq{}}\PY{l+s+s1}{There are }\PY{l+s+si}{\PYZpc{}d}\PY{l+s+s1}{ validation dog images.}\PY{l+s+s1}{\PYZsq{}} \PY{o}{\PYZpc{}} \PY{n+nb}{len}\PY{p}{(}\PY{n}{valid\PYZus{}files}\PY{p}{)}\PY{p}{)}
        \PY{n+nb}{print}\PY{p}{(}\PY{l+s+s1}{\PYZsq{}}\PY{l+s+s1}{There are }\PY{l+s+si}{\PYZpc{}d}\PY{l+s+s1}{ test dog images.}\PY{l+s+s1}{\PYZsq{}}\PY{o}{\PYZpc{}} \PY{n+nb}{len}\PY{p}{(}\PY{n}{test\PYZus{}files}\PY{p}{)}\PY{p}{)}
\end{Verbatim}


    \begin{Verbatim}[commandchars=\\\{\}]
Using TensorFlow backend.

    \end{Verbatim}

    \begin{Verbatim}[commandchars=\\\{\}]
There are 133 total dog categories.
There are 8351 total dog images.

There are 6680 training dog images.
There are 835 validation dog images.
There are 836 test dog images.

    \end{Verbatim}

    \subsubsection{Import Human Dataset}\label{import-human-dataset}

In the code cell below, we import a dataset of human images, where the
file paths are stored in the numpy array \texttt{human\_files}.

    \begin{Verbatim}[commandchars=\\\{\}]
{\color{incolor}In [{\color{incolor}2}]:} \PY{k+kn}{import} \PY{n+nn}{random}
        \PY{n}{random}\PY{o}{.}\PY{n}{seed}\PY{p}{(}\PY{l+m+mi}{8675309}\PY{p}{)}
        
        \PY{c+c1}{\PYZsh{} load filenames in shuffled human dataset}
        \PY{n}{human\PYZus{}files} \PY{o}{=} \PY{n}{np}\PY{o}{.}\PY{n}{array}\PY{p}{(}\PY{n}{glob}\PY{p}{(}\PY{l+s+s2}{\PYZdq{}}\PY{l+s+s2}{lfw/*/*}\PY{l+s+s2}{\PYZdq{}}\PY{p}{)}\PY{p}{)}
        \PY{n}{random}\PY{o}{.}\PY{n}{shuffle}\PY{p}{(}\PY{n}{human\PYZus{}files}\PY{p}{)}
        
        \PY{c+c1}{\PYZsh{} print statistics about the dataset}
        \PY{n+nb}{print}\PY{p}{(}\PY{l+s+s1}{\PYZsq{}}\PY{l+s+s1}{There are }\PY{l+s+si}{\PYZpc{}d}\PY{l+s+s1}{ total human images.}\PY{l+s+s1}{\PYZsq{}} \PY{o}{\PYZpc{}} \PY{n+nb}{len}\PY{p}{(}\PY{n}{human\PYZus{}files}\PY{p}{)}\PY{p}{)}
\end{Verbatim}


    \begin{Verbatim}[commandchars=\\\{\}]
There are 13233 total human images.

    \end{Verbatim}

    \begin{center}\rule{0.5\linewidth}{\linethickness}\end{center}

 \#\# Step 1: Detect Humans

We use OpenCV's implementation of
\href{http://docs.opencv.org/trunk/d7/d8b/tutorial_py_face_detection.html}{Haar
feature-based cascade classifiers} to detect human faces in images.
OpenCV provides many pre-trained face detectors, stored as XML files on
\href{https://github.com/opencv/opencv/tree/master/data/haarcascades}{github}.
We have downloaded one of these detectors and stored it in the
\texttt{haarcascades} directory.

In the next code cell, we demonstrate how to use this detector to find
human faces in a sample image.

    \begin{Verbatim}[commandchars=\\\{\}]
{\color{incolor}In [{\color{incolor}3}]:} \PY{k+kn}{import} \PY{n+nn}{cv2}                
        \PY{k+kn}{import} \PY{n+nn}{matplotlib}\PY{n+nn}{.}\PY{n+nn}{pyplot} \PY{k}{as} \PY{n+nn}{plt}                        
        \PY{o}{\PYZpc{}}\PY{k}{matplotlib} inline                               
        
        \PY{c+c1}{\PYZsh{} extract pre\PYZhy{}trained face detector}
        \PY{n}{face\PYZus{}cascade} \PY{o}{=} \PY{n}{cv2}\PY{o}{.}\PY{n}{CascadeClassifier}\PY{p}{(}\PY{l+s+s1}{\PYZsq{}}\PY{l+s+s1}{haarcascades/haarcascade\PYZus{}frontalface\PYZus{}alt.xml}\PY{l+s+s1}{\PYZsq{}}\PY{p}{)}
        
        \PY{c+c1}{\PYZsh{} load color (BGR) image}
        \PY{n}{img} \PY{o}{=} \PY{n}{cv2}\PY{o}{.}\PY{n}{imread}\PY{p}{(}\PY{n}{human\PYZus{}files}\PY{p}{[}\PY{l+m+mi}{3}\PY{p}{]}\PY{p}{)}
        \PY{c+c1}{\PYZsh{} convert BGR image to grayscale}
        \PY{n}{gray} \PY{o}{=} \PY{n}{cv2}\PY{o}{.}\PY{n}{cvtColor}\PY{p}{(}\PY{n}{img}\PY{p}{,} \PY{n}{cv2}\PY{o}{.}\PY{n}{COLOR\PYZus{}BGR2GRAY}\PY{p}{)}
        
        \PY{c+c1}{\PYZsh{} find faces in image}
        \PY{n}{faces} \PY{o}{=} \PY{n}{face\PYZus{}cascade}\PY{o}{.}\PY{n}{detectMultiScale}\PY{p}{(}\PY{n}{gray}\PY{p}{)}
        
        \PY{c+c1}{\PYZsh{} print number of faces detected in the image}
        \PY{n+nb}{print}\PY{p}{(}\PY{l+s+s1}{\PYZsq{}}\PY{l+s+s1}{Number of faces detected:}\PY{l+s+s1}{\PYZsq{}}\PY{p}{,} \PY{n+nb}{len}\PY{p}{(}\PY{n}{faces}\PY{p}{)}\PY{p}{)}
        
        \PY{c+c1}{\PYZsh{} get bounding box for each detected face}
        \PY{k}{for} \PY{p}{(}\PY{n}{x}\PY{p}{,}\PY{n}{y}\PY{p}{,}\PY{n}{w}\PY{p}{,}\PY{n}{h}\PY{p}{)} \PY{o+ow}{in} \PY{n}{faces}\PY{p}{:}
            \PY{c+c1}{\PYZsh{} add bounding box to color image}
            \PY{n}{cv2}\PY{o}{.}\PY{n}{rectangle}\PY{p}{(}\PY{n}{img}\PY{p}{,}\PY{p}{(}\PY{n}{x}\PY{p}{,}\PY{n}{y}\PY{p}{)}\PY{p}{,}\PY{p}{(}\PY{n}{x}\PY{o}{+}\PY{n}{w}\PY{p}{,}\PY{n}{y}\PY{o}{+}\PY{n}{h}\PY{p}{)}\PY{p}{,}\PY{p}{(}\PY{l+m+mi}{255}\PY{p}{,}\PY{l+m+mi}{0}\PY{p}{,}\PY{l+m+mi}{0}\PY{p}{)}\PY{p}{,}\PY{l+m+mi}{2}\PY{p}{)}
            
        \PY{c+c1}{\PYZsh{} convert BGR image to RGB for plotting}
        \PY{n}{cv\PYZus{}rgb} \PY{o}{=} \PY{n}{cv2}\PY{o}{.}\PY{n}{cvtColor}\PY{p}{(}\PY{n}{img}\PY{p}{,} \PY{n}{cv2}\PY{o}{.}\PY{n}{COLOR\PYZus{}BGR2RGB}\PY{p}{)}
        
        \PY{c+c1}{\PYZsh{} display the image, along with bounding box}
        \PY{n}{plt}\PY{o}{.}\PY{n}{imshow}\PY{p}{(}\PY{n}{cv\PYZus{}rgb}\PY{p}{)}
        \PY{n}{plt}\PY{o}{.}\PY{n}{show}\PY{p}{(}\PY{p}{)}
\end{Verbatim}


    \begin{Verbatim}[commandchars=\\\{\}]
('Number of faces detected:', 1)

    \end{Verbatim}

    \begin{center}
    \adjustimage{max size={0.9\linewidth}{0.9\paperheight}}{output_5_1.png}
    \end{center}
    { \hspace*{\fill} \\}
    
    Before using any of the face detectors, it is standard procedure to
convert the images to grayscale. The \texttt{detectMultiScale} function
executes the classifier stored in \texttt{face\_cascade} and takes the
grayscale image as a parameter.

In the above code, \texttt{faces} is a numpy array of detected faces,
where each row corresponds to a detected face. Each detected face is a
1D array with four entries that specifies the bounding box of the
detected face. The first two entries in the array (extracted in the
above code as \texttt{x} and \texttt{y}) specify the horizontal and
vertical positions of the top left corner of the bounding box. The last
two entries in the array (extracted here as \texttt{w} and \texttt{h})
specify the width and height of the box.

\subsubsection{Write a Human Face
Detector}\label{write-a-human-face-detector}

We can use this procedure to write a function that returns \texttt{True}
if a human face is detected in an image and \texttt{False} otherwise.
This function, aptly named \texttt{face\_detector}, takes a
string-valued file path to an image as input and appears in the code
block below.

    \begin{Verbatim}[commandchars=\\\{\}]
{\color{incolor}In [{\color{incolor}4}]:} \PY{c+c1}{\PYZsh{} returns \PYZdq{}True\PYZdq{} if face is detected in image stored at img\PYZus{}path}
        \PY{k}{def} \PY{n+nf}{face\PYZus{}detector}\PY{p}{(}\PY{n}{img\PYZus{}path}\PY{p}{)}\PY{p}{:}
            \PY{n}{img} \PY{o}{=} \PY{n}{cv2}\PY{o}{.}\PY{n}{imread}\PY{p}{(}\PY{n}{img\PYZus{}path}\PY{p}{)}
            \PY{n}{gray} \PY{o}{=} \PY{n}{cv2}\PY{o}{.}\PY{n}{cvtColor}\PY{p}{(}\PY{n}{img}\PY{p}{,} \PY{n}{cv2}\PY{o}{.}\PY{n}{COLOR\PYZus{}BGR2GRAY}\PY{p}{)}
            \PY{n}{faces} \PY{o}{=} \PY{n}{face\PYZus{}cascade}\PY{o}{.}\PY{n}{detectMultiScale}\PY{p}{(}\PY{n}{gray}\PY{p}{)}
            \PY{k}{return} \PY{n+nb}{len}\PY{p}{(}\PY{n}{faces}\PY{p}{)} \PY{o}{\PYZgt{}} \PY{l+m+mi}{0}
\end{Verbatim}


    \subsubsection{(IMPLEMENTATION) Assess the Human Face
Detector}\label{implementation-assess-the-human-face-detector}

\textbf{Question 1:} Use the code cell below to test the performance of
the \texttt{face\_detector} function.\\
- What percentage of the first 100 images in \texttt{human\_files} have
a detected human face?\\
- What percentage of the first 100 images in \texttt{dog\_files} have a
detected human face?

Ideally, we would like 100\% of human images with a detected face and
0\% of dog images with a detected face. You will see that our algorithm
falls short of this goal, but still gives acceptable performance. We
extract the file paths for the first 100 images from each of the
datasets and store them in the numpy arrays \texttt{human\_files\_short}
and \texttt{dog\_files\_short}.

\textbf{Answer:} From the first 100 images from human\_files\_short, the
face\_detector was able to identify a human face in 98 percent of all
the images. From the first 100 images from dog\_files\_short, the
face\_detector was able to identify a human face in 11 percent of all
the images.

    \begin{Verbatim}[commandchars=\\\{\}]
{\color{incolor}In [{\color{incolor}5}]:} \PY{n}{human\PYZus{}files\PYZus{}short} \PY{o}{=} \PY{n}{human\PYZus{}files}\PY{p}{[}\PY{p}{:}\PY{l+m+mi}{100}\PY{p}{]}
        \PY{n}{dog\PYZus{}files\PYZus{}short} \PY{o}{=} \PY{n}{train\PYZus{}files}\PY{p}{[}\PY{p}{:}\PY{l+m+mi}{100}\PY{p}{]}
        \PY{c+c1}{\PYZsh{} Do NOT modify the code above this line.}
        
        \PY{c+c1}{\PYZsh{}\PYZsh{} TODO: Test the performance of the face\PYZus{}detector algorithm}
        \PY{c+c1}{\PYZsh{}\PYZsh{} on the images in human\PYZus{}files\PYZus{}short and dog\PYZus{}files\PYZus{}short.}
        \PY{n}{humanCountT} \PY{o}{=} \PY{p}{[}\PY{p}{]}
        \PY{k}{for} \PY{n}{i} \PY{o+ow}{in} \PY{n+nb}{range}\PY{p}{(}\PY{n+nb}{len}\PY{p}{(}\PY{n}{human\PYZus{}files\PYZus{}short}\PY{p}{)}\PY{p}{)}\PY{p}{:}
            \PY{n}{humanCountT}\PY{o}{.}\PY{n}{append}\PY{p}{(}\PY{n}{face\PYZus{}detector}\PY{p}{(}\PY{n}{human\PYZus{}files\PYZus{}short}\PY{p}{[}\PY{n}{i}\PY{p}{]}\PY{p}{)}\PY{p}{)}
        \PY{n+nb}{print}\PY{p}{(}\PY{n+nb}{sum}\PY{p}{(}\PY{n}{humanCountT}\PY{p}{)}\PY{p}{)}
        
        \PY{n}{dogCountT} \PY{o}{=} \PY{p}{[}\PY{p}{]}
        \PY{k}{for} \PY{n}{i} \PY{o+ow}{in} \PY{n+nb}{range}\PY{p}{(}\PY{n+nb}{len}\PY{p}{(}\PY{n}{dog\PYZus{}files\PYZus{}short}\PY{p}{)}\PY{p}{)}\PY{p}{:}
            \PY{n}{dogCountT}\PY{o}{.}\PY{n}{append}\PY{p}{(}\PY{n}{face\PYZus{}detector}\PY{p}{(}\PY{n}{dog\PYZus{}files\PYZus{}short}\PY{p}{[}\PY{n}{i}\PY{p}{]}\PY{p}{)}\PY{p}{)}
        \PY{n+nb}{print}\PY{p}{(}\PY{n+nb}{sum}\PY{p}{(}\PY{n}{dogCountT}\PY{p}{)}\PY{p}{)}
\end{Verbatim}


    \begin{Verbatim}[commandchars=\\\{\}]
98
11

    \end{Verbatim}

    \textbf{Question 2:} This algorithmic choice necessitates that we
communicate to the user that we accept human images only when they
provide a clear view of a face (otherwise, we risk having unneccessarily
frustrated users!). In your opinion, is this a reasonable expectation to
pose on the user? If not, can you think of a way to detect humans in
images that does not necessitate an image with a clearly presented face?

\textbf{Answer:} I think it is reasonable to ask for the user to pose or
have a clear image of their face for the face\_detector to function
properly since most images of people are usually face towards the
camera, but Implementing a function which is capable of estimating and
completing the missing portion of face would help reduce the issue of
not catching a face. In addition, being able to detect the size ratio of
people such as the height and width can help in determining a person
from the environment.

We suggest the face detector from OpenCV as a potential way to detect
human images in your algorithm, but you are free to explore other
approaches, especially approaches that make use of deep learning :).
Please use the code cell below to design and test your own face
detection algorithm. If you decide to pursue this \emph{optional} task,
report performance on each of the datasets.

    \begin{Verbatim}[commandchars=\\\{\}]
{\color{incolor}In [{\color{incolor}6}]:} \PY{c+c1}{\PYZsh{}\PYZsh{} (Optional) TODO: Report the performance of another  }
        \PY{c+c1}{\PYZsh{}\PYZsh{} face detection algorithm on the LFW dataset}
        \PY{c+c1}{\PYZsh{}\PYZsh{}\PYZsh{} Feel free to use as many code cells as needed.}
\end{Verbatim}


    \begin{center}\rule{0.5\linewidth}{\linethickness}\end{center}

 \#\# Step 2: Detect Dogs

In this section, we use a pre-trained
\href{http://ethereon.github.io/netscope/\#/gist/db945b393d40bfa26006}{ResNet-50}
model to detect dogs in images. Our first line of code downloads the
ResNet-50 model, along with weights that have been trained on
\href{http://www.image-net.org/}{ImageNet}, a very large, very popular
dataset used for image classification and other vision tasks. ImageNet
contains over 10 million URLs, each linking to an image containing an
object from one of
\href{https://gist.github.com/yrevar/942d3a0ac09ec9e5eb3a}{1000
categories}. Given an image, this pre-trained ResNet-50 model returns a
prediction (derived from the available categories in ImageNet) for the
object that is contained in the image.

    \begin{Verbatim}[commandchars=\\\{\}]
{\color{incolor}In [{\color{incolor}7}]:} \PY{k+kn}{from} \PY{n+nn}{keras}\PY{n+nn}{.}\PY{n+nn}{applications}\PY{n+nn}{.}\PY{n+nn}{resnet50} \PY{k}{import} \PY{n}{ResNet50}
        
        \PY{c+c1}{\PYZsh{} define ResNet50 model}
        \PY{n}{ResNet50\PYZus{}model} \PY{o}{=} \PY{n}{ResNet50}\PY{p}{(}\PY{n}{weights}\PY{o}{=}\PY{l+s+s1}{\PYZsq{}}\PY{l+s+s1}{imagenet}\PY{l+s+s1}{\PYZsq{}}\PY{p}{)}
\end{Verbatim}


    \subsubsection{Pre-process the Data}\label{pre-process-the-data}

When using TensorFlow as backend, Keras CNNs require a 4D array (which
we'll also refer to as a 4D tensor) as input, with shape

\[
(\text{nb_samples}, \text{rows}, \text{columns}, \text{channels}),
\]

where \texttt{nb\_samples} corresponds to the total number of images (or
samples), and \texttt{rows}, \texttt{columns}, and \texttt{channels}
correspond to the number of rows, columns, and channels for each image,
respectively.

The \texttt{path\_to\_tensor} function below takes a string-valued file
path to a color image as input and returns a 4D tensor suitable for
supplying to a Keras CNN. The function first loads the image and resizes
it to a square image that is \(224 \times 224\) pixels. Next, the image
is converted to an array, which is then resized to a 4D tensor. In this
case, since we are working with color images, each image has three
channels. Likewise, since we are processing a single image (or sample),
the returned tensor will always have shape

\[
(1, 224, 224, 3).
\]

The \texttt{paths\_to\_tensor} function takes a numpy array of
string-valued image paths as input and returns a 4D tensor with shape

\[
(\text{nb_samples}, 224, 224, 3).
\]

Here, \texttt{nb\_samples} is the number of samples, or number of
images, in the supplied array of image paths. It is best to think of
\texttt{nb\_samples} as the number of 3D tensors (where each 3D tensor
corresponds to a different image) in your dataset!

    \begin{Verbatim}[commandchars=\\\{\}]
{\color{incolor}In [{\color{incolor}8}]:} \PY{k+kn}{from} \PY{n+nn}{keras}\PY{n+nn}{.}\PY{n+nn}{preprocessing} \PY{k}{import} \PY{n}{image}                  
        \PY{k+kn}{from} \PY{n+nn}{tqdm} \PY{k}{import} \PY{n}{tqdm}
        
        \PY{k}{def} \PY{n+nf}{path\PYZus{}to\PYZus{}tensor}\PY{p}{(}\PY{n}{img\PYZus{}path}\PY{p}{)}\PY{p}{:}
            \PY{c+c1}{\PYZsh{} loads RGB image as PIL.Image.Image type}
            \PY{n}{img} \PY{o}{=} \PY{n}{image}\PY{o}{.}\PY{n}{load\PYZus{}img}\PY{p}{(}\PY{n}{img\PYZus{}path}\PY{p}{,} \PY{n}{target\PYZus{}size}\PY{o}{=}\PY{p}{(}\PY{l+m+mi}{224}\PY{p}{,} \PY{l+m+mi}{224}\PY{p}{)}\PY{p}{)}
            \PY{c+c1}{\PYZsh{} convert PIL.Image.Image type to 3D tensor with shape (224, 224, 3)}
            \PY{n}{x} \PY{o}{=} \PY{n}{image}\PY{o}{.}\PY{n}{img\PYZus{}to\PYZus{}array}\PY{p}{(}\PY{n}{img}\PY{p}{)}
            \PY{c+c1}{\PYZsh{} convert 3D tensor to 4D tensor with shape (1, 224, 224, 3) and return 4D tensor}
            \PY{k}{return} \PY{n}{np}\PY{o}{.}\PY{n}{expand\PYZus{}dims}\PY{p}{(}\PY{n}{x}\PY{p}{,} \PY{n}{axis}\PY{o}{=}\PY{l+m+mi}{0}\PY{p}{)}
        
        \PY{k}{def} \PY{n+nf}{paths\PYZus{}to\PYZus{}tensor}\PY{p}{(}\PY{n}{img\PYZus{}paths}\PY{p}{)}\PY{p}{:}
            \PY{n}{list\PYZus{}of\PYZus{}tensors} \PY{o}{=} \PY{p}{[}\PY{n}{path\PYZus{}to\PYZus{}tensor}\PY{p}{(}\PY{n}{img\PYZus{}path}\PY{p}{)} \PY{k}{for} \PY{n}{img\PYZus{}path} \PY{o+ow}{in} \PY{n}{tqdm}\PY{p}{(}\PY{n}{img\PYZus{}paths}\PY{p}{)}\PY{p}{]}
            \PY{k}{return} \PY{n}{np}\PY{o}{.}\PY{n}{vstack}\PY{p}{(}\PY{n}{list\PYZus{}of\PYZus{}tensors}\PY{p}{)}
\end{Verbatim}


    \subsubsection{Making Predictions with
ResNet-50}\label{making-predictions-with-resnet-50}

Getting the 4D tensor ready for ResNet-50, and for any other pre-trained
model in Keras, requires some additional processing. First, the RGB
image is converted to BGR by reordering the channels. All pre-trained
models have the additional normalization step that the mean pixel
(expressed in RGB as \([103.939, 116.779, 123.68]\) and calculated from
all pixels in all images in ImageNet) must be subtracted from every
pixel in each image. This is implemented in the imported function
\texttt{preprocess\_input}. If you're curious, you can check the code
for \texttt{preprocess\_input}
\href{https://github.com/fchollet/keras/blob/master/keras/applications/imagenet_utils.py}{here}.

Now that we have a way to format our image for supplying to ResNet-50,
we are now ready to use the model to extract the predictions. This is
accomplished with the \texttt{predict} method, which returns an array
whose \(i\)-th entry is the model's predicted probability that the image
belongs to the \(i\)-th ImageNet category. This is implemented in the
\texttt{ResNet50\_predict\_labels} function below.

By taking the argmax of the predicted probability vector, we obtain an
integer corresponding to the model's predicted object class, which we
can identify with an object category through the use of this
\href{https://gist.github.com/yrevar/942d3a0ac09ec9e5eb3a}{dictionary}.

    \begin{Verbatim}[commandchars=\\\{\}]
{\color{incolor}In [{\color{incolor}9}]:} \PY{k+kn}{from} \PY{n+nn}{keras}\PY{n+nn}{.}\PY{n+nn}{applications}\PY{n+nn}{.}\PY{n+nn}{resnet50} \PY{k}{import} \PY{n}{preprocess\PYZus{}input}\PY{p}{,} \PY{n}{decode\PYZus{}predictions}
        
        \PY{k}{def} \PY{n+nf}{ResNet50\PYZus{}predict\PYZus{}labels}\PY{p}{(}\PY{n}{img\PYZus{}path}\PY{p}{)}\PY{p}{:}
            \PY{c+c1}{\PYZsh{} returns prediction vector for image located at img\PYZus{}path}
            \PY{n}{img} \PY{o}{=} \PY{n}{preprocess\PYZus{}input}\PY{p}{(}\PY{n}{path\PYZus{}to\PYZus{}tensor}\PY{p}{(}\PY{n}{img\PYZus{}path}\PY{p}{)}\PY{p}{)}
            \PY{k}{return} \PY{n}{np}\PY{o}{.}\PY{n}{argmax}\PY{p}{(}\PY{n}{ResNet50\PYZus{}model}\PY{o}{.}\PY{n}{predict}\PY{p}{(}\PY{n}{img}\PY{p}{)}\PY{p}{)}
\end{Verbatim}


    \subsubsection{Write a Dog Detector}\label{write-a-dog-detector}

While looking at the
\href{https://gist.github.com/yrevar/942d3a0ac09ec9e5eb3a}{dictionary},
you will notice that the categories corresponding to dogs appear in an
uninterrupted sequence and correspond to dictionary keys 151-268,
inclusive, to include all categories from
\texttt{\textquotesingle{}Chihuahua\textquotesingle{}} to
\texttt{\textquotesingle{}Mexican\ hairless\textquotesingle{}}. Thus, in
order to check to see if an image is predicted to contain a dog by the
pre-trained ResNet-50 model, we need only check if the
\texttt{ResNet50\_predict\_labels} function above returns a value
between 151 and 268 (inclusive).

We use these ideas to complete the \texttt{dog\_detector} function
below, which returns \texttt{True} if a dog is detected in an image (and
\texttt{False} if not).

    \begin{Verbatim}[commandchars=\\\{\}]
{\color{incolor}In [{\color{incolor}10}]:} \PY{c+c1}{\PYZsh{}\PYZsh{}\PYZsh{} returns \PYZdq{}True\PYZdq{} if a dog is detected in the image stored at img\PYZus{}path}
         \PY{k}{def} \PY{n+nf}{dog\PYZus{}detector}\PY{p}{(}\PY{n}{img\PYZus{}path}\PY{p}{)}\PY{p}{:}
             \PY{n}{prediction} \PY{o}{=} \PY{n}{ResNet50\PYZus{}predict\PYZus{}labels}\PY{p}{(}\PY{n}{img\PYZus{}path}\PY{p}{)}
             \PY{k}{return} \PY{p}{(}\PY{p}{(}\PY{n}{prediction} \PY{o}{\PYZlt{}}\PY{o}{=} \PY{l+m+mi}{268}\PY{p}{)} \PY{o}{\PYZam{}} \PY{p}{(}\PY{n}{prediction} \PY{o}{\PYZgt{}}\PY{o}{=} \PY{l+m+mi}{151}\PY{p}{)}\PY{p}{)} 
\end{Verbatim}


    \subsubsection{(IMPLEMENTATION) Assess the Dog
Detector}\label{implementation-assess-the-dog-detector}

\textbf{Question 3:} Use the code cell below to test the performance of
your \texttt{dog\_detector} function.\\
- What percentage of the images in \texttt{human\_files\_short} have a
detected dog?\\
- What percentage of the images in \texttt{dog\_files\_short} have a
detected dog?

\textbf{Answer:} From the first 100 images from human\_files\_short, the
dog\_detector was able to identify a dog breed in 2 percent of all the
images. From the first 100 images from dog\_files\_short, the
dog\_detector was able to identify a dog breed in 100 percent of all the
images.

    \begin{Verbatim}[commandchars=\\\{\}]
{\color{incolor}In [{\color{incolor}11}]:} \PY{c+c1}{\PYZsh{}\PYZsh{}\PYZsh{} TODO: Test the performance of the dog\PYZus{}detector function}
         \PY{c+c1}{\PYZsh{}\PYZsh{}\PYZsh{} on the images in human\PYZus{}files\PYZus{}short and dog\PYZus{}files\PYZus{}short.}
         \PY{n}{humanCountT} \PY{o}{=} \PY{p}{[}\PY{p}{]}
         \PY{k}{for} \PY{n}{i} \PY{o+ow}{in} \PY{n+nb}{range}\PY{p}{(}\PY{n+nb}{len}\PY{p}{(}\PY{n}{human\PYZus{}files\PYZus{}short}\PY{p}{)}\PY{p}{)}\PY{p}{:}
             \PY{n}{humanCountT}\PY{o}{.}\PY{n}{append}\PY{p}{(}\PY{n}{dog\PYZus{}detector}\PY{p}{(}\PY{n}{human\PYZus{}files\PYZus{}short}\PY{p}{[}\PY{n}{i}\PY{p}{]}\PY{p}{)}\PY{p}{)}
         \PY{n+nb}{print}\PY{p}{(}\PY{n+nb}{sum}\PY{p}{(}\PY{n}{humanCountT}\PY{p}{)}\PY{p}{)}
         
         \PY{n}{dogCountT} \PY{o}{=} \PY{p}{[}\PY{p}{]}
         \PY{k}{for} \PY{n}{i} \PY{o+ow}{in} \PY{n+nb}{range}\PY{p}{(}\PY{n+nb}{len}\PY{p}{(}\PY{n}{dog\PYZus{}files\PYZus{}short}\PY{p}{)}\PY{p}{)}\PY{p}{:}
             \PY{n}{dogCountT}\PY{o}{.}\PY{n}{append}\PY{p}{(}\PY{n}{dog\PYZus{}detector}\PY{p}{(}\PY{n}{dog\PYZus{}files\PYZus{}short}\PY{p}{[}\PY{n}{i}\PY{p}{]}\PY{p}{)}\PY{p}{)}
         \PY{n+nb}{print}\PY{p}{(}\PY{n+nb}{sum}\PY{p}{(}\PY{n}{dogCountT}\PY{p}{)}\PY{p}{)}
\end{Verbatim}


    \begin{Verbatim}[commandchars=\\\{\}]
2
100

    \end{Verbatim}

    \begin{center}\rule{0.5\linewidth}{\linethickness}\end{center}

 \#\# Step 3: Create a CNN to Classify Dog Breeds (from Scratch)

Now that we have functions for detecting humans and dogs in images, we
need a way to predict breed from images. In this step, you will create a
CNN that classifies dog breeds. You must create your CNN \emph{from
scratch} (so, you can't use transfer learning \emph{yet}!), and you must
attain a test accuracy of at least 1\%. In Step 5 of this notebook, you
will have the opportunity to use transfer learning to create a CNN that
attains greatly improved accuracy.

Be careful with adding too many trainable layers! More parameters means
longer training, which means you are more likely to need a GPU to
accelerate the training process. Thankfully, Keras provides a handy
estimate of the time that each epoch is likely to take; you can
extrapolate this estimate to figure out how long it will take for your
algorithm to train.

We mention that the task of assigning breed to dogs from images is
considered exceptionally challenging. To see why, consider that
\emph{even a human} would have great difficulty in distinguishing
between a Brittany and a Welsh Springer Spaniel.

\begin{longtable}[]{@{}ll@{}}
\toprule
Brittany & Welsh Springer Spaniel\tabularnewline
\midrule
\endhead
&\tabularnewline
\bottomrule
\end{longtable}

It is not difficult to find other dog breed pairs with minimal
inter-class variation (for instance, Curly-Coated Retrievers and
American Water Spaniels).

\begin{longtable}[]{@{}ll@{}}
\toprule
Curly-Coated Retriever & American Water Spaniel\tabularnewline
\midrule
\endhead
&\tabularnewline
\bottomrule
\end{longtable}

Likewise, recall that labradors come in yellow, chocolate, and black.
Your vision-based algorithm will have to conquer this high intra-class
variation to determine how to classify all of these different shades as
the same breed.

\begin{longtable}[]{@{}ll@{}}
\toprule
Yellow Labrador & Chocolate Labrador\tabularnewline
\midrule
\endhead
&\tabularnewline
\bottomrule
\end{longtable}

We also mention that random chance presents an exceptionally low bar:
setting aside the fact that the classes are slightly imabalanced, a
random guess will provide a correct answer roughly 1 in 133 times, which
corresponds to an accuracy of less than 1\%.

Remember that the practice is far ahead of the theory in deep learning.
Experiment with many different architectures, and trust your intuition.
And, of course, have fun!

\subsubsection{Pre-process the Data}\label{pre-process-the-data}

We rescale the images by dividing every pixel in every image by 255.

    \begin{Verbatim}[commandchars=\\\{\}]
{\color{incolor}In [{\color{incolor}12}]:} \PY{k+kn}{from} \PY{n+nn}{PIL} \PY{k}{import} \PY{n}{ImageFile}                            
         \PY{n}{ImageFile}\PY{o}{.}\PY{n}{LOAD\PYZus{}TRUNCATED\PYZus{}IMAGES} \PY{o}{=} \PY{k+kc}{True}                 
         
         \PY{c+c1}{\PYZsh{} pre\PYZhy{}process the data for Keras}
         \PY{n}{train\PYZus{}tensors} \PY{o}{=} \PY{n}{paths\PYZus{}to\PYZus{}tensor}\PY{p}{(}\PY{n}{train\PYZus{}files}\PY{p}{)}\PY{o}{.}\PY{n}{astype}\PY{p}{(}\PY{l+s+s1}{\PYZsq{}}\PY{l+s+s1}{float32}\PY{l+s+s1}{\PYZsq{}}\PY{p}{)}\PY{o}{/}\PY{l+m+mi}{255}
         \PY{n}{valid\PYZus{}tensors} \PY{o}{=} \PY{n}{paths\PYZus{}to\PYZus{}tensor}\PY{p}{(}\PY{n}{valid\PYZus{}files}\PY{p}{)}\PY{o}{.}\PY{n}{astype}\PY{p}{(}\PY{l+s+s1}{\PYZsq{}}\PY{l+s+s1}{float32}\PY{l+s+s1}{\PYZsq{}}\PY{p}{)}\PY{o}{/}\PY{l+m+mi}{255}
         \PY{n}{test\PYZus{}tensors} \PY{o}{=} \PY{n}{paths\PYZus{}to\PYZus{}tensor}\PY{p}{(}\PY{n}{test\PYZus{}files}\PY{p}{)}\PY{o}{.}\PY{n}{astype}\PY{p}{(}\PY{l+s+s1}{\PYZsq{}}\PY{l+s+s1}{float32}\PY{l+s+s1}{\PYZsq{}}\PY{p}{)}\PY{o}{/}\PY{l+m+mi}{255}
\end{Verbatim}


    \begin{Verbatim}[commandchars=\\\{\}]
100\%|██████████| 6680/6680 [00:50<00:00, 133.23it/s]
100\%|██████████| 835/835 [00:05<00:00, 147.77it/s]
100\%|██████████| 836/836 [00:05<00:00, 148.86it/s]

    \end{Verbatim}

    \subsubsection{(IMPLEMENTATION) Model
Architecture}\label{implementation-model-architecture}

Create a CNN to classify dog breed. At the end of your code cell block,
summarize the layers of your model by executing the line:

\begin{verbatim}
    model.summary()
\end{verbatim}

We have imported some Python modules to get you started, but feel free
to import as many modules as you need. If you end up getting stuck,
here's a hint that specifies a model that trains relatively fast on CPU
and attains \textgreater{}1\% test accuracy in 5 epochs:

\begin{figure}
\centering
\includegraphics{images/sample_cnn.png}
\caption{Sample CNN}
\end{figure}

\textbf{Question 4:} Outline the steps you took to get to your final CNN
architecture and your reasoning at each step. If you chose to use the
hinted architecture above, describe why you think that CNN architecture
should work well for the image classification task.

\textbf{Answer:} I did try running different architectures but none of
them performed better than the provided architecture. The reason the
provided architecture performed well is the low parameter count which
allows for key elements to have a higher weighted value.

    \begin{Verbatim}[commandchars=\\\{\}]
{\color{incolor}In [{\color{incolor}71}]:} \PY{k+kn}{from} \PY{n+nn}{keras}\PY{n+nn}{.}\PY{n+nn}{layers} \PY{k}{import} \PY{n}{Conv2D}\PY{p}{,} \PY{n}{MaxPooling2D}\PY{p}{,} \PY{n}{GlobalAveragePooling2D}
         \PY{k+kn}{from} \PY{n+nn}{keras}\PY{n+nn}{.}\PY{n+nn}{layers} \PY{k}{import} \PY{n}{Dropout}\PY{p}{,} \PY{n}{Flatten}\PY{p}{,} \PY{n}{Dense}
         \PY{k+kn}{from} \PY{n+nn}{keras}\PY{n+nn}{.}\PY{n+nn}{models} \PY{k}{import} \PY{n}{Sequential}
         
         \PY{n}{model} \PY{o}{=} \PY{n}{Sequential}\PY{p}{(}\PY{p}{)}
         
         \PY{c+c1}{\PYZsh{}\PYZsh{}\PYZsh{} TODO: Define your architecture.}
         
         \PY{n}{model}\PY{o}{.}\PY{n}{add}\PY{p}{(}\PY{n}{Conv2D}\PY{p}{(}\PY{n}{filters}\PY{o}{=}\PY{l+m+mi}{16}\PY{p}{,} \PY{n}{kernel\PYZus{}size}\PY{o}{=}\PY{l+m+mi}{2}\PY{p}{,} \PY{n}{padding}\PY{o}{=}\PY{l+s+s1}{\PYZsq{}}\PY{l+s+s1}{same}\PY{l+s+s1}{\PYZsq{}}\PY{p}{,} \PY{n}{activation}\PY{o}{=}\PY{l+s+s1}{\PYZsq{}}\PY{l+s+s1}{relu}\PY{l+s+s1}{\PYZsq{}}\PY{p}{,} 
                                 \PY{n}{input\PYZus{}shape}\PY{o}{=}\PY{p}{(}\PY{l+m+mi}{224}\PY{p}{,} \PY{l+m+mi}{224}\PY{p}{,} \PY{l+m+mi}{3}\PY{p}{)}\PY{p}{)}\PY{p}{)}
         
         \PY{n}{model}\PY{o}{.}\PY{n}{add}\PY{p}{(}\PY{n}{MaxPooling2D}\PY{p}{(}\PY{n}{pool\PYZus{}size}\PY{o}{=}\PY{l+m+mi}{2}\PY{p}{)}\PY{p}{)}
         
         \PY{n}{model}\PY{o}{.}\PY{n}{add}\PY{p}{(}\PY{n}{Conv2D}\PY{p}{(}\PY{n}{filters}\PY{o}{=}\PY{l+m+mi}{32}\PY{p}{,} \PY{n}{kernel\PYZus{}size}\PY{o}{=}\PY{l+m+mi}{2}\PY{p}{,} \PY{n}{padding}\PY{o}{=}\PY{l+s+s1}{\PYZsq{}}\PY{l+s+s1}{same}\PY{l+s+s1}{\PYZsq{}}\PY{p}{,} \PY{n}{activation}\PY{o}{=}\PY{l+s+s1}{\PYZsq{}}\PY{l+s+s1}{relu}\PY{l+s+s1}{\PYZsq{}}\PY{p}{)}\PY{p}{)}
         \PY{n}{model}\PY{o}{.}\PY{n}{add}\PY{p}{(}\PY{n}{MaxPooling2D}\PY{p}{(}\PY{n}{pool\PYZus{}size}\PY{o}{=}\PY{l+m+mi}{2}\PY{p}{)}\PY{p}{)}
         
         \PY{n}{model}\PY{o}{.}\PY{n}{add}\PY{p}{(}\PY{n}{Conv2D}\PY{p}{(}\PY{n}{filters}\PY{o}{=}\PY{l+m+mi}{64}\PY{p}{,} \PY{n}{kernel\PYZus{}size}\PY{o}{=}\PY{l+m+mi}{2}\PY{p}{,} \PY{n}{padding}\PY{o}{=}\PY{l+s+s1}{\PYZsq{}}\PY{l+s+s1}{same}\PY{l+s+s1}{\PYZsq{}}\PY{p}{,} \PY{n}{activation}\PY{o}{=}\PY{l+s+s1}{\PYZsq{}}\PY{l+s+s1}{relu}\PY{l+s+s1}{\PYZsq{}}\PY{p}{)}\PY{p}{)}
         \PY{n}{model}\PY{o}{.}\PY{n}{add}\PY{p}{(}\PY{n}{MaxPooling2D}\PY{p}{(}\PY{n}{pool\PYZus{}size}\PY{o}{=}\PY{l+m+mi}{2}\PY{p}{)}\PY{p}{)}
         
         \PY{c+c1}{\PYZsh{}model.add(Dropout(0.3))}
         \PY{n}{model}\PY{o}{.}\PY{n}{add}\PY{p}{(}\PY{n}{GlobalAveragePooling2D}\PY{p}{(}\PY{p}{)}\PY{p}{)}
         \PY{c+c1}{\PYZsh{}model.add(Flatten())}
         \PY{c+c1}{\PYZsh{}model.add(Dense(133, activation=\PYZsq{}relu\PYZsq{}))}
         \PY{c+c1}{\PYZsh{}model.add(Dropout(0.4))}
         \PY{n}{model}\PY{o}{.}\PY{n}{add}\PY{p}{(}\PY{n}{Dense}\PY{p}{(}\PY{l+m+mi}{133}\PY{p}{,} \PY{n}{activation}\PY{o}{=}\PY{l+s+s1}{\PYZsq{}}\PY{l+s+s1}{softmax}\PY{l+s+s1}{\PYZsq{}}\PY{p}{)}\PY{p}{)}
         
         \PY{n}{model}\PY{o}{.}\PY{n}{summary}\PY{p}{(}\PY{p}{)}
\end{Verbatim}


    \begin{Verbatim}[commandchars=\\\{\}]
\_\_\_\_\_\_\_\_\_\_\_\_\_\_\_\_\_\_\_\_\_\_\_\_\_\_\_\_\_\_\_\_\_\_\_\_\_\_\_\_\_\_\_\_\_\_\_\_\_\_\_\_\_\_\_\_\_\_\_\_\_\_\_\_\_
Layer (type)                 Output Shape              Param \#   
=================================================================
conv2d\_10 (Conv2D)           (None, 224, 224, 16)      208       
\_\_\_\_\_\_\_\_\_\_\_\_\_\_\_\_\_\_\_\_\_\_\_\_\_\_\_\_\_\_\_\_\_\_\_\_\_\_\_\_\_\_\_\_\_\_\_\_\_\_\_\_\_\_\_\_\_\_\_\_\_\_\_\_\_
max\_pooling2d\_5 (MaxPooling2 (None, 112, 112, 16)      0         
\_\_\_\_\_\_\_\_\_\_\_\_\_\_\_\_\_\_\_\_\_\_\_\_\_\_\_\_\_\_\_\_\_\_\_\_\_\_\_\_\_\_\_\_\_\_\_\_\_\_\_\_\_\_\_\_\_\_\_\_\_\_\_\_\_
conv2d\_11 (Conv2D)           (None, 112, 112, 32)      2080      
\_\_\_\_\_\_\_\_\_\_\_\_\_\_\_\_\_\_\_\_\_\_\_\_\_\_\_\_\_\_\_\_\_\_\_\_\_\_\_\_\_\_\_\_\_\_\_\_\_\_\_\_\_\_\_\_\_\_\_\_\_\_\_\_\_
max\_pooling2d\_6 (MaxPooling2 (None, 56, 56, 32)        0         
\_\_\_\_\_\_\_\_\_\_\_\_\_\_\_\_\_\_\_\_\_\_\_\_\_\_\_\_\_\_\_\_\_\_\_\_\_\_\_\_\_\_\_\_\_\_\_\_\_\_\_\_\_\_\_\_\_\_\_\_\_\_\_\_\_
conv2d\_12 (Conv2D)           (None, 56, 56, 64)        8256      
\_\_\_\_\_\_\_\_\_\_\_\_\_\_\_\_\_\_\_\_\_\_\_\_\_\_\_\_\_\_\_\_\_\_\_\_\_\_\_\_\_\_\_\_\_\_\_\_\_\_\_\_\_\_\_\_\_\_\_\_\_\_\_\_\_
max\_pooling2d\_7 (MaxPooling2 (None, 28, 28, 64)        0         
\_\_\_\_\_\_\_\_\_\_\_\_\_\_\_\_\_\_\_\_\_\_\_\_\_\_\_\_\_\_\_\_\_\_\_\_\_\_\_\_\_\_\_\_\_\_\_\_\_\_\_\_\_\_\_\_\_\_\_\_\_\_\_\_\_
global\_average\_pooling2d\_5 ( (None, 64)                0         
\_\_\_\_\_\_\_\_\_\_\_\_\_\_\_\_\_\_\_\_\_\_\_\_\_\_\_\_\_\_\_\_\_\_\_\_\_\_\_\_\_\_\_\_\_\_\_\_\_\_\_\_\_\_\_\_\_\_\_\_\_\_\_\_\_
dense\_5 (Dense)              (None, 133)               8645      
=================================================================
Total params: 19,189
Trainable params: 19,189
Non-trainable params: 0
\_\_\_\_\_\_\_\_\_\_\_\_\_\_\_\_\_\_\_\_\_\_\_\_\_\_\_\_\_\_\_\_\_\_\_\_\_\_\_\_\_\_\_\_\_\_\_\_\_\_\_\_\_\_\_\_\_\_\_\_\_\_\_\_\_

    \end{Verbatim}

    \subsubsection{Compile the Model}\label{compile-the-model}

    \begin{Verbatim}[commandchars=\\\{\}]
{\color{incolor}In [{\color{incolor}14}]:} \PY{n}{model}\PY{o}{.}\PY{n}{compile}\PY{p}{(}\PY{n}{optimizer}\PY{o}{=}\PY{l+s+s1}{\PYZsq{}}\PY{l+s+s1}{rmsprop}\PY{l+s+s1}{\PYZsq{}}\PY{p}{,} \PY{n}{loss}\PY{o}{=}\PY{l+s+s1}{\PYZsq{}}\PY{l+s+s1}{categorical\PYZus{}crossentropy}\PY{l+s+s1}{\PYZsq{}}\PY{p}{,} \PY{n}{metrics}\PY{o}{=}\PY{p}{[}\PY{l+s+s1}{\PYZsq{}}\PY{l+s+s1}{accuracy}\PY{l+s+s1}{\PYZsq{}}\PY{p}{]}\PY{p}{)}
\end{Verbatim}


    \subsubsection{(IMPLEMENTATION) Train the
Model}\label{implementation-train-the-model}

Train your model in the code cell below. Use model checkpointing to save
the model that attains the best validation loss.

You are welcome to
\href{https://blog.keras.io/building-powerful-image-classification-models-using-very-little-data.html}{augment
the training data}, but this is not a requirement.

    \begin{Verbatim}[commandchars=\\\{\}]
{\color{incolor}In [{\color{incolor}15}]:} \PY{k+kn}{from} \PY{n+nn}{keras}\PY{n+nn}{.}\PY{n+nn}{callbacks} \PY{k}{import} \PY{n}{ModelCheckpoint}  
         
         \PY{c+c1}{\PYZsh{}\PYZsh{}\PYZsh{} TODO: specify the number of epochs that you would like to use to train the model.}
         
         \PY{n}{epochs} \PY{o}{=} \PY{l+m+mi}{30}
         
         \PY{c+c1}{\PYZsh{}\PYZsh{}\PYZsh{} Do NOT modify the code below this line.}
         
         \PY{n}{checkpointer} \PY{o}{=} \PY{n}{ModelCheckpoint}\PY{p}{(}\PY{n}{filepath}\PY{o}{=}\PY{l+s+s1}{\PYZsq{}}\PY{l+s+s1}{saved\PYZus{}models/weights.best.from\PYZus{}scratch.hdf5}\PY{l+s+s1}{\PYZsq{}}\PY{p}{,} 
                                        \PY{n}{verbose}\PY{o}{=}\PY{l+m+mi}{1}\PY{p}{,} \PY{n}{save\PYZus{}best\PYZus{}only}\PY{o}{=}\PY{k+kc}{True}\PY{p}{)}
         
         \PY{n}{model}\PY{o}{.}\PY{n}{fit}\PY{p}{(}\PY{n}{train\PYZus{}tensors}\PY{p}{,} \PY{n}{train\PYZus{}targets}\PY{p}{,} 
                   \PY{n}{validation\PYZus{}data}\PY{o}{=}\PY{p}{(}\PY{n}{valid\PYZus{}tensors}\PY{p}{,} \PY{n}{valid\PYZus{}targets}\PY{p}{)}\PY{p}{,}
                   \PY{n}{epochs}\PY{o}{=}\PY{n}{epochs}\PY{p}{,} \PY{n}{batch\PYZus{}size}\PY{o}{=}\PY{l+m+mi}{20}\PY{p}{,} \PY{n}{callbacks}\PY{o}{=}\PY{p}{[}\PY{n}{checkpointer}\PY{p}{]}\PY{p}{,} \PY{n}{verbose}\PY{o}{=}\PY{l+m+mi}{1}\PY{p}{)}
\end{Verbatim}


    \begin{Verbatim}[commandchars=\\\{\}]
Train on 6680 samples, validate on 835 samples
Epoch 1/30
6660/6680 [============================>.] - ETA: 0s - loss: 4.8837 - acc: 0.0090Epoch 00000: val\_loss improved from inf to 4.87022, saving model to saved\_models/weights.best.from\_scratch.hdf5
6680/6680 [==============================] - 21s - loss: 4.8837 - acc: 0.0090 - val\_loss: 4.8702 - val\_acc: 0.0108
Epoch 2/30
6660/6680 [============================>.] - ETA: 0s - loss: 4.8621 - acc: 0.0123Epoch 00001: val\_loss improved from 4.87022 to 4.84494, saving model to saved\_models/weights.best.from\_scratch.hdf5
6680/6680 [==============================] - 20s - loss: 4.8620 - acc: 0.0123 - val\_loss: 4.8449 - val\_acc: 0.0180
Epoch 3/30
6660/6680 [============================>.] - ETA: 0s - loss: 4.8200 - acc: 0.0165Epoch 00002: val\_loss improved from 4.84494 to 4.81525, saving model to saved\_models/weights.best.from\_scratch.hdf5
6680/6680 [==============================] - 20s - loss: 4.8198 - acc: 0.0168 - val\_loss: 4.8152 - val\_acc: 0.0180
Epoch 4/30
6660/6680 [============================>.] - ETA: 0s - loss: 4.7857 - acc: 0.0200Epoch 00003: val\_loss improved from 4.81525 to 4.78880, saving model to saved\_models/weights.best.from\_scratch.hdf5
6680/6680 [==============================] - 20s - loss: 4.7856 - acc: 0.0201 - val\_loss: 4.7888 - val\_acc: 0.0240
Epoch 5/30
6660/6680 [============================>.] - ETA: 0s - loss: 4.7598 - acc: 0.0204Epoch 00004: val\_loss did not improve
6680/6680 [==============================] - 20s - loss: 4.7595 - acc: 0.0204 - val\_loss: 4.7936 - val\_acc: 0.0156
Epoch 6/30
6660/6680 [============================>.] - ETA: 0s - loss: 4.7297 - acc: 0.0237Epoch 00005: val\_loss improved from 4.78880 to 4.74668, saving model to saved\_models/weights.best.from\_scratch.hdf5
6680/6680 [==============================] - 20s - loss: 4.7292 - acc: 0.0238 - val\_loss: 4.7467 - val\_acc: 0.0228
Epoch 7/30
6660/6680 [============================>.] - ETA: 0s - loss: 4.7021 - acc: 0.0302Epoch 00006: val\_loss improved from 4.74668 to 4.73691, saving model to saved\_models/weights.best.from\_scratch.hdf5
6680/6680 [==============================] - 20s - loss: 4.7025 - acc: 0.0301 - val\_loss: 4.7369 - val\_acc: 0.0192
Epoch 8/30
6660/6680 [============================>.] - ETA: 0s - loss: 4.6695 - acc: 0.0281Epoch 00007: val\_loss improved from 4.73691 to 4.72329, saving model to saved\_models/weights.best.from\_scratch.hdf5
6680/6680 [==============================] - 20s - loss: 4.6701 - acc: 0.0281 - val\_loss: 4.7233 - val\_acc: 0.0263
Epoch 9/30
6660/6680 [============================>.] - ETA: 0s - loss: 4.6451 - acc: 0.0335Epoch 00008: val\_loss improved from 4.72329 to 4.70884, saving model to saved\_models/weights.best.from\_scratch.hdf5
6680/6680 [==============================] - 20s - loss: 4.6444 - acc: 0.0334 - val\_loss: 4.7088 - val\_acc: 0.0335
Epoch 10/30
6660/6680 [============================>.] - ETA: 0s - loss: 4.6167 - acc: 0.0386Epoch 00009: val\_loss improved from 4.70884 to 4.67466, saving model to saved\_models/weights.best.from\_scratch.hdf5
6680/6680 [==============================] - 20s - loss: 4.6167 - acc: 0.0388 - val\_loss: 4.6747 - val\_acc: 0.0347
Epoch 11/30
6660/6680 [============================>.] - ETA: 0s - loss: 4.5954 - acc: 0.0350Epoch 00010: val\_loss improved from 4.67466 to 4.66157, saving model to saved\_models/weights.best.from\_scratch.hdf5
6680/6680 [==============================] - 20s - loss: 4.5951 - acc: 0.0350 - val\_loss: 4.6616 - val\_acc: 0.0335
Epoch 12/30
6660/6680 [============================>.] - ETA: 0s - loss: 4.5687 - acc: 0.0410Epoch 00011: val\_loss improved from 4.66157 to 4.64982, saving model to saved\_models/weights.best.from\_scratch.hdf5
6680/6680 [==============================] - 20s - loss: 4.5685 - acc: 0.0412 - val\_loss: 4.6498 - val\_acc: 0.0311
Epoch 13/30
6660/6680 [============================>.] - ETA: 0s - loss: 4.5507 - acc: 0.0455Epoch 00012: val\_loss improved from 4.64982 to 4.62401, saving model to saved\_models/weights.best.from\_scratch.hdf5
6680/6680 [==============================] - 20s - loss: 4.5512 - acc: 0.0454 - val\_loss: 4.6240 - val\_acc: 0.0287
Epoch 14/30
6660/6680 [============================>.] - ETA: 0s - loss: 4.5294 - acc: 0.0449Epoch 00013: val\_loss improved from 4.62401 to 4.62111, saving model to saved\_models/weights.best.from\_scratch.hdf5
6680/6680 [==============================] - 20s - loss: 4.5294 - acc: 0.0448 - val\_loss: 4.6211 - val\_acc: 0.0335
Epoch 15/30
6660/6680 [============================>.] - ETA: 0s - loss: 4.5039 - acc: 0.0470Epoch 00014: val\_loss improved from 4.62111 to 4.60749, saving model to saved\_models/weights.best.from\_scratch.hdf5
6680/6680 [==============================] - 20s - loss: 4.5044 - acc: 0.0470 - val\_loss: 4.6075 - val\_acc: 0.0419
Epoch 16/30
6660/6680 [============================>.] - ETA: 0s - loss: 4.4842 - acc: 0.0494Epoch 00015: val\_loss improved from 4.60749 to 4.58038, saving model to saved\_models/weights.best.from\_scratch.hdf5
6680/6680 [==============================] - 20s - loss: 4.4849 - acc: 0.0493 - val\_loss: 4.5804 - val\_acc: 0.0407
Epoch 17/30
6660/6680 [============================>.] - ETA: 0s - loss: 4.4705 - acc: 0.0508Epoch 00016: val\_loss did not improve
6680/6680 [==============================] - 20s - loss: 4.4705 - acc: 0.0507 - val\_loss: 4.5862 - val\_acc: 0.0347
Epoch 18/30
6660/6680 [============================>.] - ETA: 0s - loss: 4.4484 - acc: 0.0551Epoch 00017: val\_loss did not improve
6680/6680 [==============================] - 20s - loss: 4.4479 - acc: 0.0549 - val\_loss: 4.5876 - val\_acc: 0.0359
Epoch 19/30
6660/6680 [============================>.] - ETA: 0s - loss: 4.4317 - acc: 0.0568Epoch 00018: val\_loss improved from 4.58038 to 4.55891, saving model to saved\_models/weights.best.from\_scratch.hdf5
6680/6680 [==============================] - 20s - loss: 4.4312 - acc: 0.0569 - val\_loss: 4.5589 - val\_acc: 0.0467
Epoch 20/30
6660/6680 [============================>.] - ETA: 0s - loss: 4.4140 - acc: 0.0560Epoch 00019: val\_loss improved from 4.55891 to 4.53984, saving model to saved\_models/weights.best.from\_scratch.hdf5
6680/6680 [==============================] - 20s - loss: 4.4136 - acc: 0.0560 - val\_loss: 4.5398 - val\_acc: 0.0383
Epoch 21/30
6660/6680 [============================>.] - ETA: 0s - loss: 4.3895 - acc: 0.0578Epoch 00020: val\_loss improved from 4.53984 to 4.53363, saving model to saved\_models/weights.best.from\_scratch.hdf5
6680/6680 [==============================] - 20s - loss: 4.3896 - acc: 0.0576 - val\_loss: 4.5336 - val\_acc: 0.0347
Epoch 22/30
6660/6680 [============================>.] - ETA: 0s - loss: 4.3738 - acc: 0.0616Epoch 00021: val\_loss did not improve
6680/6680 [==============================] - 20s - loss: 4.3725 - acc: 0.0620 - val\_loss: 4.5454 - val\_acc: 0.0371
Epoch 23/30
6660/6680 [============================>.] - ETA: 0s - loss: 4.3531 - acc: 0.0616Epoch 00022: val\_loss improved from 4.53363 to 4.50766, saving model to saved\_models/weights.best.from\_scratch.hdf5
6680/6680 [==============================] - 20s - loss: 4.3534 - acc: 0.0614 - val\_loss: 4.5077 - val\_acc: 0.0419
Epoch 24/30
6660/6680 [============================>.] - ETA: 0s - loss: 4.3358 - acc: 0.0667Epoch 00023: val\_loss improved from 4.50766 to 4.49363, saving model to saved\_models/weights.best.from\_scratch.hdf5
6680/6680 [==============================] - 20s - loss: 4.3365 - acc: 0.0669 - val\_loss: 4.4936 - val\_acc: 0.0407
Epoch 25/30
6660/6680 [============================>.] - ETA: 0s - loss: 4.3147 - acc: 0.0704Epoch 00024: val\_loss improved from 4.49363 to 4.46196, saving model to saved\_models/weights.best.from\_scratch.hdf5
6680/6680 [==============================] - 20s - loss: 4.3157 - acc: 0.0705 - val\_loss: 4.4620 - val\_acc: 0.0455
Epoch 26/30
6660/6680 [============================>.] - ETA: 0s - loss: 4.2948 - acc: 0.0712Epoch 00025: val\_loss improved from 4.46196 to 4.45913, saving model to saved\_models/weights.best.from\_scratch.hdf5
6680/6680 [==============================] - 20s - loss: 4.2950 - acc: 0.0710 - val\_loss: 4.4591 - val\_acc: 0.0467
Epoch 27/30
6660/6680 [============================>.] - ETA: 0s - loss: 4.2744 - acc: 0.0718Epoch 00026: val\_loss improved from 4.45913 to 4.42987, saving model to saved\_models/weights.best.from\_scratch.hdf5
6680/6680 [==============================] - 20s - loss: 4.2750 - acc: 0.0717 - val\_loss: 4.4299 - val\_acc: 0.0431
Epoch 28/30
6660/6680 [============================>.] - ETA: 0s - loss: 4.2560 - acc: 0.0737Epoch 00027: val\_loss did not improve
6680/6680 [==============================] - 20s - loss: 4.2562 - acc: 0.0737 - val\_loss: 4.4558 - val\_acc: 0.0443
Epoch 29/30
6660/6680 [============================>.] - ETA: 0s - loss: 4.2345 - acc: 0.0784Epoch 00028: val\_loss improved from 4.42987 to 4.41328, saving model to saved\_models/weights.best.from\_scratch.hdf5
6680/6680 [==============================] - 20s - loss: 4.2351 - acc: 0.0786 - val\_loss: 4.4133 - val\_acc: 0.0503
Epoch 30/30
6660/6680 [============================>.] - ETA: 0s - loss: 4.2170 - acc: 0.0773Epoch 00029: val\_loss did not improve
6680/6680 [==============================] - 20s - loss: 4.2165 - acc: 0.0774 - val\_loss: 4.4260 - val\_acc: 0.0539

    \end{Verbatim}

\begin{Verbatim}[commandchars=\\\{\}]
{\color{outcolor}Out[{\color{outcolor}15}]:} <keras.callbacks.History at 0x7f259405c690>
\end{Verbatim}
            
    \subsubsection{Load the Model with the Best Validation
Loss}\label{load-the-model-with-the-best-validation-loss}

    \begin{Verbatim}[commandchars=\\\{\}]
{\color{incolor}In [{\color{incolor}16}]:} \PY{n}{model}\PY{o}{.}\PY{n}{load\PYZus{}weights}\PY{p}{(}\PY{l+s+s1}{\PYZsq{}}\PY{l+s+s1}{saved\PYZus{}models/weights.best.from\PYZus{}scratch.hdf5}\PY{l+s+s1}{\PYZsq{}}\PY{p}{)}
         \PY{n}{model}\PY{o}{.}\PY{n}{summary}\PY{p}{(}\PY{p}{)}
\end{Verbatim}


    \begin{Verbatim}[commandchars=\\\{\}]
\_\_\_\_\_\_\_\_\_\_\_\_\_\_\_\_\_\_\_\_\_\_\_\_\_\_\_\_\_\_\_\_\_\_\_\_\_\_\_\_\_\_\_\_\_\_\_\_\_\_\_\_\_\_\_\_\_\_\_\_\_\_\_\_\_
Layer (type)                 Output Shape              Param \#   
=================================================================
conv2d\_1 (Conv2D)            (None, 224, 224, 16)      208       
\_\_\_\_\_\_\_\_\_\_\_\_\_\_\_\_\_\_\_\_\_\_\_\_\_\_\_\_\_\_\_\_\_\_\_\_\_\_\_\_\_\_\_\_\_\_\_\_\_\_\_\_\_\_\_\_\_\_\_\_\_\_\_\_\_
max\_pooling2d\_2 (MaxPooling2 (None, 112, 112, 16)      0         
\_\_\_\_\_\_\_\_\_\_\_\_\_\_\_\_\_\_\_\_\_\_\_\_\_\_\_\_\_\_\_\_\_\_\_\_\_\_\_\_\_\_\_\_\_\_\_\_\_\_\_\_\_\_\_\_\_\_\_\_\_\_\_\_\_
conv2d\_2 (Conv2D)            (None, 112, 112, 32)      2080      
\_\_\_\_\_\_\_\_\_\_\_\_\_\_\_\_\_\_\_\_\_\_\_\_\_\_\_\_\_\_\_\_\_\_\_\_\_\_\_\_\_\_\_\_\_\_\_\_\_\_\_\_\_\_\_\_\_\_\_\_\_\_\_\_\_
max\_pooling2d\_3 (MaxPooling2 (None, 56, 56, 32)        0         
\_\_\_\_\_\_\_\_\_\_\_\_\_\_\_\_\_\_\_\_\_\_\_\_\_\_\_\_\_\_\_\_\_\_\_\_\_\_\_\_\_\_\_\_\_\_\_\_\_\_\_\_\_\_\_\_\_\_\_\_\_\_\_\_\_
conv2d\_3 (Conv2D)            (None, 56, 56, 64)        8256      
\_\_\_\_\_\_\_\_\_\_\_\_\_\_\_\_\_\_\_\_\_\_\_\_\_\_\_\_\_\_\_\_\_\_\_\_\_\_\_\_\_\_\_\_\_\_\_\_\_\_\_\_\_\_\_\_\_\_\_\_\_\_\_\_\_
max\_pooling2d\_4 (MaxPooling2 (None, 28, 28, 64)        0         
\_\_\_\_\_\_\_\_\_\_\_\_\_\_\_\_\_\_\_\_\_\_\_\_\_\_\_\_\_\_\_\_\_\_\_\_\_\_\_\_\_\_\_\_\_\_\_\_\_\_\_\_\_\_\_\_\_\_\_\_\_\_\_\_\_
global\_average\_pooling2d\_1 ( (None, 64)                0         
\_\_\_\_\_\_\_\_\_\_\_\_\_\_\_\_\_\_\_\_\_\_\_\_\_\_\_\_\_\_\_\_\_\_\_\_\_\_\_\_\_\_\_\_\_\_\_\_\_\_\_\_\_\_\_\_\_\_\_\_\_\_\_\_\_
dense\_1 (Dense)              (None, 133)               8645      
=================================================================
Total params: 19,189
Trainable params: 19,189
Non-trainable params: 0
\_\_\_\_\_\_\_\_\_\_\_\_\_\_\_\_\_\_\_\_\_\_\_\_\_\_\_\_\_\_\_\_\_\_\_\_\_\_\_\_\_\_\_\_\_\_\_\_\_\_\_\_\_\_\_\_\_\_\_\_\_\_\_\_\_

    \end{Verbatim}

    \subsubsection{Test the Model}\label{test-the-model}

Try out your model on the test dataset of dog images. Ensure that your
test accuracy is greater than 1\%.

    \begin{Verbatim}[commandchars=\\\{\}]
{\color{incolor}In [{\color{incolor}17}]:} \PY{c+c1}{\PYZsh{} get index of predicted dog breed for each image in test set}
         \PY{n}{dog\PYZus{}breed\PYZus{}predictions} \PY{o}{=} \PY{p}{[}\PY{n}{np}\PY{o}{.}\PY{n}{argmax}\PY{p}{(}\PY{n}{model}\PY{o}{.}\PY{n}{predict}\PY{p}{(}\PY{n}{np}\PY{o}{.}\PY{n}{expand\PYZus{}dims}\PY{p}{(}\PY{n}{tensor}\PY{p}{,} \PY{n}{axis}\PY{o}{=}\PY{l+m+mi}{0}\PY{p}{)}\PY{p}{)}\PY{p}{)} \PY{k}{for} \PY{n}{tensor} \PY{o+ow}{in} \PY{n}{test\PYZus{}tensors}\PY{p}{]}
         
         \PY{c+c1}{\PYZsh{} report test accuracy}
         \PY{n}{test\PYZus{}accuracy} \PY{o}{=} \PY{l+m+mi}{100}\PY{o}{*}\PY{n}{np}\PY{o}{.}\PY{n}{sum}\PY{p}{(}\PY{n}{np}\PY{o}{.}\PY{n}{array}\PY{p}{(}\PY{n}{dog\PYZus{}breed\PYZus{}predictions}\PY{p}{)}\PY{o}{==}\PY{n}{np}\PY{o}{.}\PY{n}{argmax}\PY{p}{(}\PY{n}{test\PYZus{}targets}\PY{p}{,} \PY{n}{axis}\PY{o}{=}\PY{l+m+mi}{1}\PY{p}{)}\PY{p}{)}\PY{o}{/}\PY{n+nb}{len}\PY{p}{(}\PY{n}{dog\PYZus{}breed\PYZus{}predictions}\PY{p}{)}
         \PY{n+nb}{print}\PY{p}{(}\PY{l+s+s1}{\PYZsq{}}\PY{l+s+s1}{Test accuracy: }\PY{l+s+si}{\PYZpc{}.4f}\PY{l+s+si}{\PYZpc{}\PYZpc{}}\PY{l+s+s1}{\PYZsq{}} \PY{o}{\PYZpc{}} \PY{n}{test\PYZus{}accuracy}\PY{p}{)}
\end{Verbatim}


    \begin{Verbatim}[commandchars=\\\{\}]
Test accuracy: 7.0000\%

    \end{Verbatim}

    \begin{center}\rule{0.5\linewidth}{\linethickness}\end{center}

 \#\# Step 4: Use a CNN to Classify Dog Breeds

To reduce training time without sacrificing accuracy, we show you how to
train a CNN using transfer learning. In the following step, you will get
a chance to use transfer learning to train your own CNN.

\subsubsection{Obtain Bottleneck
Features}\label{obtain-bottleneck-features}

    \begin{Verbatim}[commandchars=\\\{\}]
{\color{incolor}In [{\color{incolor}18}]:} \PY{n}{bottleneck\PYZus{}features} \PY{o}{=} \PY{n}{np}\PY{o}{.}\PY{n}{load}\PY{p}{(}\PY{l+s+s1}{\PYZsq{}}\PY{l+s+s1}{bottleneck\PYZus{}features/DogVGG16Data.npz}\PY{l+s+s1}{\PYZsq{}}\PY{p}{)}
         \PY{n}{train\PYZus{}VGG16} \PY{o}{=} \PY{n}{bottleneck\PYZus{}features}\PY{p}{[}\PY{l+s+s1}{\PYZsq{}}\PY{l+s+s1}{train}\PY{l+s+s1}{\PYZsq{}}\PY{p}{]}
         \PY{n}{valid\PYZus{}VGG16} \PY{o}{=} \PY{n}{bottleneck\PYZus{}features}\PY{p}{[}\PY{l+s+s1}{\PYZsq{}}\PY{l+s+s1}{valid}\PY{l+s+s1}{\PYZsq{}}\PY{p}{]}
         \PY{n}{test\PYZus{}VGG16} \PY{o}{=} \PY{n}{bottleneck\PYZus{}features}\PY{p}{[}\PY{l+s+s1}{\PYZsq{}}\PY{l+s+s1}{test}\PY{l+s+s1}{\PYZsq{}}\PY{p}{]}
\end{Verbatim}


    \subsubsection{Model Architecture}\label{model-architecture}

The model uses the the pre-trained VGG-16 model as a fixed feature
extractor, where the last convolutional output of VGG-16 is fed as input
to our model. We only add a global average pooling layer and a fully
connected layer, where the latter contains one node for each dog
category and is equipped with a softmax.

    \begin{Verbatim}[commandchars=\\\{\}]
{\color{incolor}In [{\color{incolor}19}]:} \PY{n}{VGG16\PYZus{}model} \PY{o}{=} \PY{n}{Sequential}\PY{p}{(}\PY{p}{)}
         \PY{n}{VGG16\PYZus{}model}\PY{o}{.}\PY{n}{add}\PY{p}{(}\PY{n}{GlobalAveragePooling2D}\PY{p}{(}\PY{n}{input\PYZus{}shape}\PY{o}{=}\PY{n}{train\PYZus{}VGG16}\PY{o}{.}\PY{n}{shape}\PY{p}{[}\PY{l+m+mi}{1}\PY{p}{:}\PY{p}{]}\PY{p}{)}\PY{p}{)}
         \PY{n}{VGG16\PYZus{}model}\PY{o}{.}\PY{n}{add}\PY{p}{(}\PY{n}{Dense}\PY{p}{(}\PY{l+m+mi}{133}\PY{p}{,} \PY{n}{activation}\PY{o}{=}\PY{l+s+s1}{\PYZsq{}}\PY{l+s+s1}{softmax}\PY{l+s+s1}{\PYZsq{}}\PY{p}{)}\PY{p}{)}
         
         \PY{n}{VGG16\PYZus{}model}\PY{o}{.}\PY{n}{summary}\PY{p}{(}\PY{p}{)}
\end{Verbatim}


    \begin{Verbatim}[commandchars=\\\{\}]
\_\_\_\_\_\_\_\_\_\_\_\_\_\_\_\_\_\_\_\_\_\_\_\_\_\_\_\_\_\_\_\_\_\_\_\_\_\_\_\_\_\_\_\_\_\_\_\_\_\_\_\_\_\_\_\_\_\_\_\_\_\_\_\_\_
Layer (type)                 Output Shape              Param \#   
=================================================================
global\_average\_pooling2d\_2 ( (None, 512)               0         
\_\_\_\_\_\_\_\_\_\_\_\_\_\_\_\_\_\_\_\_\_\_\_\_\_\_\_\_\_\_\_\_\_\_\_\_\_\_\_\_\_\_\_\_\_\_\_\_\_\_\_\_\_\_\_\_\_\_\_\_\_\_\_\_\_
dense\_2 (Dense)              (None, 133)               68229     
=================================================================
Total params: 68,229
Trainable params: 68,229
Non-trainable params: 0
\_\_\_\_\_\_\_\_\_\_\_\_\_\_\_\_\_\_\_\_\_\_\_\_\_\_\_\_\_\_\_\_\_\_\_\_\_\_\_\_\_\_\_\_\_\_\_\_\_\_\_\_\_\_\_\_\_\_\_\_\_\_\_\_\_

    \end{Verbatim}

    \subsubsection{Compile the Model}\label{compile-the-model}

    \begin{Verbatim}[commandchars=\\\{\}]
{\color{incolor}In [{\color{incolor}20}]:} \PY{n}{VGG16\PYZus{}model}\PY{o}{.}\PY{n}{compile}\PY{p}{(}\PY{n}{loss}\PY{o}{=}\PY{l+s+s1}{\PYZsq{}}\PY{l+s+s1}{categorical\PYZus{}crossentropy}\PY{l+s+s1}{\PYZsq{}}\PY{p}{,} \PY{n}{optimizer}\PY{o}{=}\PY{l+s+s1}{\PYZsq{}}\PY{l+s+s1}{rmsprop}\PY{l+s+s1}{\PYZsq{}}\PY{p}{,} \PY{n}{metrics}\PY{o}{=}\PY{p}{[}\PY{l+s+s1}{\PYZsq{}}\PY{l+s+s1}{accuracy}\PY{l+s+s1}{\PYZsq{}}\PY{p}{]}\PY{p}{)}
\end{Verbatim}


    \subsubsection{Train the Model}\label{train-the-model}

    \begin{Verbatim}[commandchars=\\\{\}]
{\color{incolor}In [{\color{incolor}21}]:} \PY{n}{checkpointer} \PY{o}{=} \PY{n}{ModelCheckpoint}\PY{p}{(}\PY{n}{filepath}\PY{o}{=}\PY{l+s+s1}{\PYZsq{}}\PY{l+s+s1}{saved\PYZus{}models/weights.best.VGG16.hdf5}\PY{l+s+s1}{\PYZsq{}}\PY{p}{,} 
                                        \PY{n}{verbose}\PY{o}{=}\PY{l+m+mi}{1}\PY{p}{,} \PY{n}{save\PYZus{}best\PYZus{}only}\PY{o}{=}\PY{k+kc}{True}\PY{p}{)}
         
         \PY{n}{VGG16\PYZus{}model}\PY{o}{.}\PY{n}{fit}\PY{p}{(}\PY{n}{train\PYZus{}VGG16}\PY{p}{,} \PY{n}{train\PYZus{}targets}\PY{p}{,} 
                   \PY{n}{validation\PYZus{}data}\PY{o}{=}\PY{p}{(}\PY{n}{valid\PYZus{}VGG16}\PY{p}{,} \PY{n}{valid\PYZus{}targets}\PY{p}{)}\PY{p}{,}
                   \PY{n}{epochs}\PY{o}{=}\PY{l+m+mi}{20}\PY{p}{,} \PY{n}{batch\PYZus{}size}\PY{o}{=}\PY{l+m+mi}{20}\PY{p}{,} \PY{n}{callbacks}\PY{o}{=}\PY{p}{[}\PY{n}{checkpointer}\PY{p}{]}\PY{p}{,} \PY{n}{verbose}\PY{o}{=}\PY{l+m+mi}{1}\PY{p}{)}
\end{Verbatim}


    \begin{Verbatim}[commandchars=\\\{\}]
Train on 6680 samples, validate on 835 samples
Epoch 1/20
6620/6680 [============================>.] - ETA: 0s - loss: 12.0821 - acc: 0.1341Epoch 00000: val\_loss improved from inf to 10.57644, saving model to saved\_models/weights.best.VGG16.hdf5
6680/6680 [==============================] - 1s - loss: 12.0704 - acc: 0.1344 - val\_loss: 10.5764 - val\_acc: 0.2156
Epoch 2/20
6640/6680 [============================>.] - ETA: 0s - loss: 9.8870 - acc: 0.2922Epoch 00001: val\_loss improved from 10.57644 to 9.93543, saving model to saved\_models/weights.best.VGG16.hdf5
6680/6680 [==============================] - 1s - loss: 9.8875 - acc: 0.2924 - val\_loss: 9.9354 - val\_acc: 0.2994
Epoch 3/20
6640/6680 [============================>.] - ETA: 0s - loss: 9.3890 - acc: 0.3542Epoch 00002: val\_loss improved from 9.93543 to 9.62314, saving model to saved\_models/weights.best.VGG16.hdf5
6680/6680 [==============================] - 1s - loss: 9.3918 - acc: 0.3536 - val\_loss: 9.6231 - val\_acc: 0.3102
Epoch 4/20
6660/6680 [============================>.] - ETA: 0s - loss: 9.1355 - acc: 0.3839Epoch 00003: val\_loss improved from 9.62314 to 9.52626, saving model to saved\_models/weights.best.VGG16.hdf5
6680/6680 [==============================] - 1s - loss: 9.1297 - acc: 0.3843 - val\_loss: 9.5263 - val\_acc: 0.3257
Epoch 5/20
6620/6680 [============================>.] - ETA: 0s - loss: 8.9084 - acc: 0.4080Epoch 00004: val\_loss improved from 9.52626 to 9.30172, saving model to saved\_models/weights.best.VGG16.hdf5
6680/6680 [==============================] - 1s - loss: 8.9236 - acc: 0.4072 - val\_loss: 9.3017 - val\_acc: 0.3545
Epoch 6/20
6660/6680 [============================>.] - ETA: 0s - loss: 8.7901 - acc: 0.4254Epoch 00005: val\_loss did not improve
6680/6680 [==============================] - 1s - loss: 8.7879 - acc: 0.4256 - val\_loss: 9.3192 - val\_acc: 0.3509
Epoch 7/20
6620/6680 [============================>.] - ETA: 0s - loss: 8.7439 - acc: 0.4370Epoch 00006: val\_loss did not improve
6680/6680 [==============================] - 1s - loss: 8.7304 - acc: 0.4376 - val\_loss: 9.3106 - val\_acc: 0.3569
Epoch 8/20
6600/6680 [============================>.] - ETA: 0s - loss: 8.6736 - acc: 0.4444Epoch 00007: val\_loss improved from 9.30172 to 9.12046, saving model to saved\_models/weights.best.VGG16.hdf5
6680/6680 [==============================] - 1s - loss: 8.6825 - acc: 0.4437 - val\_loss: 9.1205 - val\_acc: 0.3725
Epoch 9/20
6440/6680 [===========================>..] - ETA: 0s - loss: 8.5621 - acc: 0.4568Epoch 00008: val\_loss improved from 9.12046 to 9.10076, saving model to saved\_models/weights.best.VGG16.hdf5
6680/6680 [==============================] - 1s - loss: 8.5718 - acc: 0.4563 - val\_loss: 9.1008 - val\_acc: 0.3760
Epoch 10/20
6420/6680 [===========================>..] - ETA: 0s - loss: 8.5430 - acc: 0.4569Epoch 00009: val\_loss did not improve
6680/6680 [==============================] - 1s - loss: 8.5446 - acc: 0.4569 - val\_loss: 9.1189 - val\_acc: 0.3760
Epoch 11/20
6620/6680 [============================>.] - ETA: 0s - loss: 8.4819 - acc: 0.4636Epoch 00010: val\_loss improved from 9.10076 to 9.08437, saving model to saved\_models/weights.best.VGG16.hdf5
6680/6680 [==============================] - 1s - loss: 8.4952 - acc: 0.4627 - val\_loss: 9.0844 - val\_acc: 0.3749
Epoch 12/20
6520/6680 [============================>.] - ETA: 0s - loss: 8.4953 - acc: 0.4658Epoch 00011: val\_loss improved from 9.08437 to 9.04504, saving model to saved\_models/weights.best.VGG16.hdf5
6680/6680 [==============================] - 1s - loss: 8.4760 - acc: 0.4669 - val\_loss: 9.0450 - val\_acc: 0.3820
Epoch 13/20
6600/6680 [============================>.] - ETA: 0s - loss: 8.4844 - acc: 0.4670Epoch 00012: val\_loss improved from 9.04504 to 9.02596, saving model to saved\_models/weights.best.VGG16.hdf5
6680/6680 [==============================] - 1s - loss: 8.4673 - acc: 0.4681 - val\_loss: 9.0260 - val\_acc: 0.3916
Epoch 14/20
6520/6680 [============================>.] - ETA: 0s - loss: 8.4563 - acc: 0.4704Epoch 00013: val\_loss did not improve
6680/6680 [==============================] - 1s - loss: 8.4448 - acc: 0.4711 - val\_loss: 9.0414 - val\_acc: 0.3844
Epoch 15/20
6620/6680 [============================>.] - ETA: 0s - loss: 8.3740 - acc: 0.4718Epoch 00014: val\_loss improved from 9.02596 to 8.88652, saving model to saved\_models/weights.best.VGG16.hdf5
6680/6680 [==============================] - 1s - loss: 8.3763 - acc: 0.4716 - val\_loss: 8.8865 - val\_acc: 0.3928
Epoch 16/20
6560/6680 [============================>.] - ETA: 0s - loss: 8.1686 - acc: 0.4780Epoch 00015: val\_loss improved from 8.88652 to 8.70889, saving model to saved\_models/weights.best.VGG16.hdf5
6680/6680 [==============================] - 1s - loss: 8.1623 - acc: 0.4777 - val\_loss: 8.7089 - val\_acc: 0.4012
Epoch 17/20
6580/6680 [============================>.] - ETA: 0s - loss: 7.9720 - acc: 0.4956Epoch 00016: val\_loss improved from 8.70889 to 8.53480, saving model to saved\_models/weights.best.VGG16.hdf5
6680/6680 [==============================] - 1s - loss: 7.9594 - acc: 0.4963 - val\_loss: 8.5348 - val\_acc: 0.4048
Epoch 18/20
6620/6680 [============================>.] - ETA: 0s - loss: 7.7834 - acc: 0.5036Epoch 00017: val\_loss improved from 8.53480 to 8.33110, saving model to saved\_models/weights.best.VGG16.hdf5
6680/6680 [==============================] - 1s - loss: 7.7793 - acc: 0.5039 - val\_loss: 8.3311 - val\_acc: 0.4000
Epoch 19/20
6520/6680 [============================>.] - ETA: 0s - loss: 7.5254 - acc: 0.5130Epoch 00018: val\_loss improved from 8.33110 to 8.21812, saving model to saved\_models/weights.best.VGG16.hdf5
6680/6680 [==============================] - 1s - loss: 7.5189 - acc: 0.5133 - val\_loss: 8.2181 - val\_acc: 0.4180
Epoch 20/20
6560/6680 [============================>.] - ETA: 0s - loss: 7.3702 - acc: 0.5274Epoch 00019: val\_loss improved from 8.21812 to 8.08787, saving model to saved\_models/weights.best.VGG16.hdf5
6680/6680 [==============================] - 1s - loss: 7.3563 - acc: 0.5284 - val\_loss: 8.0879 - val\_acc: 0.4323

    \end{Verbatim}

\begin{Verbatim}[commandchars=\\\{\}]
{\color{outcolor}Out[{\color{outcolor}21}]:} <keras.callbacks.History at 0x7f257842f0d0>
\end{Verbatim}
            
    \subsubsection{Load the Model with the Best Validation
Loss}\label{load-the-model-with-the-best-validation-loss}

    \begin{Verbatim}[commandchars=\\\{\}]
{\color{incolor}In [{\color{incolor}22}]:} \PY{n}{VGG16\PYZus{}model}\PY{o}{.}\PY{n}{load\PYZus{}weights}\PY{p}{(}\PY{l+s+s1}{\PYZsq{}}\PY{l+s+s1}{saved\PYZus{}models/weights.best.VGG16.hdf5}\PY{l+s+s1}{\PYZsq{}}\PY{p}{)}
         \PY{n}{VGG16\PYZus{}model}\PY{o}{.}\PY{n}{summary}\PY{p}{(}\PY{p}{)}
\end{Verbatim}


    \begin{Verbatim}[commandchars=\\\{\}]
\_\_\_\_\_\_\_\_\_\_\_\_\_\_\_\_\_\_\_\_\_\_\_\_\_\_\_\_\_\_\_\_\_\_\_\_\_\_\_\_\_\_\_\_\_\_\_\_\_\_\_\_\_\_\_\_\_\_\_\_\_\_\_\_\_
Layer (type)                 Output Shape              Param \#   
=================================================================
global\_average\_pooling2d\_2 ( (None, 512)               0         
\_\_\_\_\_\_\_\_\_\_\_\_\_\_\_\_\_\_\_\_\_\_\_\_\_\_\_\_\_\_\_\_\_\_\_\_\_\_\_\_\_\_\_\_\_\_\_\_\_\_\_\_\_\_\_\_\_\_\_\_\_\_\_\_\_
dense\_2 (Dense)              (None, 133)               68229     
=================================================================
Total params: 68,229
Trainable params: 68,229
Non-trainable params: 0
\_\_\_\_\_\_\_\_\_\_\_\_\_\_\_\_\_\_\_\_\_\_\_\_\_\_\_\_\_\_\_\_\_\_\_\_\_\_\_\_\_\_\_\_\_\_\_\_\_\_\_\_\_\_\_\_\_\_\_\_\_\_\_\_\_

    \end{Verbatim}

    \subsubsection{Test the Model}\label{test-the-model}

Now, we can use the CNN to test how well it identifies breed within our
test dataset of dog images. We print the test accuracy below.

    \begin{Verbatim}[commandchars=\\\{\}]
{\color{incolor}In [{\color{incolor}23}]:} \PY{c+c1}{\PYZsh{} get index of predicted dog breed for each image in test set}
         \PY{n}{VGG16\PYZus{}predictions} \PY{o}{=} \PY{p}{[}\PY{n}{np}\PY{o}{.}\PY{n}{argmax}\PY{p}{(}\PY{n}{VGG16\PYZus{}model}\PY{o}{.}\PY{n}{predict}\PY{p}{(}\PY{n}{np}\PY{o}{.}\PY{n}{expand\PYZus{}dims}\PY{p}{(}\PY{n}{feature}\PY{p}{,} \PY{n}{axis}\PY{o}{=}\PY{l+m+mi}{0}\PY{p}{)}\PY{p}{)}\PY{p}{)} \PY{k}{for} \PY{n}{feature} \PY{o+ow}{in} \PY{n}{test\PYZus{}VGG16}\PY{p}{]}
         
         \PY{c+c1}{\PYZsh{} report test accuracy}
         \PY{n}{test\PYZus{}accuracy} \PY{o}{=} \PY{l+m+mi}{100}\PY{o}{*}\PY{n}{np}\PY{o}{.}\PY{n}{sum}\PY{p}{(}\PY{n}{np}\PY{o}{.}\PY{n}{array}\PY{p}{(}\PY{n}{VGG16\PYZus{}predictions}\PY{p}{)}\PY{o}{==}\PY{n}{np}\PY{o}{.}\PY{n}{argmax}\PY{p}{(}\PY{n}{test\PYZus{}targets}\PY{p}{,} \PY{n}{axis}\PY{o}{=}\PY{l+m+mi}{1}\PY{p}{)}\PY{p}{)}\PY{o}{/}\PY{n+nb}{len}\PY{p}{(}\PY{n}{VGG16\PYZus{}predictions}\PY{p}{)}
         \PY{n+nb}{print}\PY{p}{(}\PY{l+s+s1}{\PYZsq{}}\PY{l+s+s1}{Test accuracy: }\PY{l+s+si}{\PYZpc{}.4f}\PY{l+s+si}{\PYZpc{}\PYZpc{}}\PY{l+s+s1}{\PYZsq{}} \PY{o}{\PYZpc{}} \PY{n}{test\PYZus{}accuracy}\PY{p}{)}
\end{Verbatim}


    \begin{Verbatim}[commandchars=\\\{\}]
Test accuracy: 44.0000\%

    \end{Verbatim}

    \subsubsection{Predict Dog Breed with the
Model}\label{predict-dog-breed-with-the-model}

    \begin{Verbatim}[commandchars=\\\{\}]
{\color{incolor}In [{\color{incolor}24}]:} \PY{k+kn}{from} \PY{n+nn}{extract\PYZus{}bottleneck\PYZus{}features} \PY{k}{import} \PY{o}{*}
         
         \PY{k}{def} \PY{n+nf}{VGG16\PYZus{}predict\PYZus{}breed}\PY{p}{(}\PY{n}{img\PYZus{}path}\PY{p}{)}\PY{p}{:}
             \PY{c+c1}{\PYZsh{} extract bottleneck features}
             \PY{n}{bottleneck\PYZus{}feature} \PY{o}{=} \PY{n}{extract\PYZus{}VGG16}\PY{p}{(}\PY{n}{path\PYZus{}to\PYZus{}tensor}\PY{p}{(}\PY{n}{img\PYZus{}path}\PY{p}{)}\PY{p}{)}
             \PY{c+c1}{\PYZsh{} obtain predicted vector}
             \PY{n}{predicted\PYZus{}vector} \PY{o}{=} \PY{n}{VGG16\PYZus{}model}\PY{o}{.}\PY{n}{predict}\PY{p}{(}\PY{n}{bottleneck\PYZus{}feature}\PY{p}{)}
             \PY{c+c1}{\PYZsh{} return dog breed that is predicted by the model}
             \PY{k}{return} \PY{n}{dog\PYZus{}names}\PY{p}{[}\PY{n}{np}\PY{o}{.}\PY{n}{argmax}\PY{p}{(}\PY{n}{predicted\PYZus{}vector}\PY{p}{)}\PY{p}{]}
\end{Verbatim}


    \begin{center}\rule{0.5\linewidth}{\linethickness}\end{center}

 \#\# Step 5: Create a CNN to Classify Dog Breeds (using Transfer
Learning)

You will now use transfer learning to create a CNN that can identify dog
breed from images. Your CNN must attain at least 60\% accuracy on the
test set.

In Step 4, we used transfer learning to create a CNN using VGG-16
bottleneck features. In this section, you must use the bottleneck
features from a different pre-trained model. To make things easier for
you, we have pre-computed the features for all of the networks that are
currently available in Keras: -
\href{https://s3-us-west-1.amazonaws.com/udacity-aind/dog-project/DogVGG19Data.npz}{VGG-19}
bottleneck features -
\href{https://s3-us-west-1.amazonaws.com/udacity-aind/dog-project/DogResnet50Data.npz}{ResNet-50}
bottleneck features -
\href{https://s3-us-west-1.amazonaws.com/udacity-aind/dog-project/DogInceptionV3Data.npz}{Inception}
bottleneck features -
\href{https://s3-us-west-1.amazonaws.com/udacity-aind/dog-project/DogXceptionData.npz}{Xception}
bottleneck features

The files are encoded as such:

\begin{verbatim}
Dog{network}Data.npz
\end{verbatim}

where \texttt{\{network\}}, in the above filename, can be one of
\texttt{VGG19}, \texttt{Resnet50}, \texttt{InceptionV3}, or
\texttt{Xception}. Pick one of the above architectures, download the
corresponding bottleneck features, and store the downloaded file in the
\texttt{bottleneck\_features/} folder in the repository.

\subsubsection{(IMPLEMENTATION) Obtain Bottleneck
Features}\label{implementation-obtain-bottleneck-features}

In the code block below, extract the bottleneck features corresponding
to the train, test, and validation sets by running the following:

\begin{verbatim}
bottleneck_features = np.load('bottleneck_features/Dog{network}Data.npz')
train_{network} = bottleneck_features['train']
valid_{network} = bottleneck_features['valid']
test_{network} = bottleneck_features['test']
\end{verbatim}

    \begin{Verbatim}[commandchars=\\\{\}]
{\color{incolor}In [{\color{incolor}72}]:} \PY{c+c1}{\PYZsh{}\PYZsh{}\PYZsh{} TODO: Obtain bottleneck features from another pre\PYZhy{}trained CNN.}
         \PY{n}{bottleneck\PYZus{}features} \PY{o}{=} \PY{n}{np}\PY{o}{.}\PY{n}{load}\PY{p}{(}\PY{l+s+s1}{\PYZsq{}}\PY{l+s+s1}{bottleneck\PYZus{}features/DogVGG19Data.npz}\PY{l+s+s1}{\PYZsq{}}\PY{p}{)}
         \PY{n}{train\PYZus{}VGG19} \PY{o}{=} \PY{n}{bottleneck\PYZus{}features}\PY{p}{[}\PY{l+s+s1}{\PYZsq{}}\PY{l+s+s1}{train}\PY{l+s+s1}{\PYZsq{}}\PY{p}{]}
         \PY{n}{valid\PYZus{}VGG19} \PY{o}{=} \PY{n}{bottleneck\PYZus{}features}\PY{p}{[}\PY{l+s+s1}{\PYZsq{}}\PY{l+s+s1}{valid}\PY{l+s+s1}{\PYZsq{}}\PY{p}{]}
         \PY{n}{test\PYZus{}VGG19} \PY{o}{=} \PY{n}{bottleneck\PYZus{}features}\PY{p}{[}\PY{l+s+s1}{\PYZsq{}}\PY{l+s+s1}{test}\PY{l+s+s1}{\PYZsq{}}\PY{p}{]}
\end{Verbatim}


    \subsubsection{(IMPLEMENTATION) Model
Architecture}\label{implementation-model-architecture}

Create a CNN to classify dog breed. At the end of your code cell block,
summarize the layers of your model by executing the line:

\begin{verbatim}
    <your model's name>.summary()
\end{verbatim}

\textbf{Question 5:} Outline the steps you took to get to your final CNN
architecture and your reasoning at each step. Describe why you think the
architecture is suitable for the current problem.

\textbf{Answer:} I decided to first implement a similar architecture
from step 3 but it resulted in very poor results, so I decided to alter
the architecture by removing different layers. I obtained the best
results when I removed the dropout layers in the architecture. The
architecture is suitable in my opinion, but I do think dropout layers
are important to remove any outliers in the model and increase the
accuracy. Ideally, I would like to have an architecture close to 85\% to
have a more reliable accuracy.

    \begin{Verbatim}[commandchars=\\\{\}]
{\color{incolor}In [{\color{incolor}73}]:} \PY{c+c1}{\PYZsh{}\PYZsh{}\PYZsh{} TODO: Define your architecture.}
         \PY{n}{VGG19\PYZus{}model} \PY{o}{=} \PY{n}{Sequential}\PY{p}{(}\PY{p}{)}
         
         
         \PY{n}{VGG19\PYZus{}model}\PY{o}{.}\PY{n}{add}\PY{p}{(}\PY{n}{Conv2D}\PY{p}{(}\PY{n}{filters}\PY{o}{=}\PY{l+m+mi}{16}\PY{p}{,} \PY{n}{kernel\PYZus{}size}\PY{o}{=}\PY{l+m+mi}{2}\PY{p}{,} \PY{n}{padding}\PY{o}{=}\PY{l+s+s1}{\PYZsq{}}\PY{l+s+s1}{same}\PY{l+s+s1}{\PYZsq{}}\PY{p}{,} \PY{n}{activation}\PY{o}{=}\PY{l+s+s1}{\PYZsq{}}\PY{l+s+s1}{relu}\PY{l+s+s1}{\PYZsq{}}\PY{p}{,} 
                                 \PY{n}{input\PYZus{}shape}\PY{o}{=}\PY{n}{train\PYZus{}VGG19}\PY{o}{.}\PY{n}{shape}\PY{p}{[}\PY{l+m+mi}{1}\PY{p}{:}\PY{p}{]}\PY{p}{)}\PY{p}{)}
         
         \PY{n}{VGG19\PYZus{}model}\PY{o}{.}\PY{n}{add}\PY{p}{(}\PY{n}{Conv2D}\PY{p}{(}\PY{n}{filters}\PY{o}{=}\PY{l+m+mi}{32}\PY{p}{,} \PY{n}{kernel\PYZus{}size}\PY{o}{=}\PY{l+m+mi}{2}\PY{p}{,} \PY{n}{padding}\PY{o}{=}\PY{l+s+s1}{\PYZsq{}}\PY{l+s+s1}{same}\PY{l+s+s1}{\PYZsq{}}\PY{p}{,} \PY{n}{activation}\PY{o}{=}\PY{l+s+s1}{\PYZsq{}}\PY{l+s+s1}{relu}\PY{l+s+s1}{\PYZsq{}}\PY{p}{)}\PY{p}{)}
         
         
         \PY{n}{VGG19\PYZus{}model}\PY{o}{.}\PY{n}{add}\PY{p}{(}\PY{n}{GlobalAveragePooling2D}\PY{p}{(}\PY{p}{)}\PY{p}{)}
         
         \PY{n}{VGG19\PYZus{}model}\PY{o}{.}\PY{n}{add}\PY{p}{(}\PY{n}{Dense}\PY{p}{(}\PY{l+m+mi}{133}\PY{p}{,} \PY{n}{activation}\PY{o}{=}\PY{l+s+s1}{\PYZsq{}}\PY{l+s+s1}{softmax}\PY{l+s+s1}{\PYZsq{}}\PY{p}{)}\PY{p}{)}
         
         
         
         \PY{n}{VGG19\PYZus{}model}\PY{o}{.}\PY{n}{summary}\PY{p}{(}\PY{p}{)}
\end{Verbatim}


    \begin{Verbatim}[commandchars=\\\{\}]
\_\_\_\_\_\_\_\_\_\_\_\_\_\_\_\_\_\_\_\_\_\_\_\_\_\_\_\_\_\_\_\_\_\_\_\_\_\_\_\_\_\_\_\_\_\_\_\_\_\_\_\_\_\_\_\_\_\_\_\_\_\_\_\_\_
Layer (type)                 Output Shape              Param \#   
=================================================================
conv2d\_13 (Conv2D)           (None, 7, 7, 16)          32784     
\_\_\_\_\_\_\_\_\_\_\_\_\_\_\_\_\_\_\_\_\_\_\_\_\_\_\_\_\_\_\_\_\_\_\_\_\_\_\_\_\_\_\_\_\_\_\_\_\_\_\_\_\_\_\_\_\_\_\_\_\_\_\_\_\_
conv2d\_14 (Conv2D)           (None, 7, 7, 32)          2080      
\_\_\_\_\_\_\_\_\_\_\_\_\_\_\_\_\_\_\_\_\_\_\_\_\_\_\_\_\_\_\_\_\_\_\_\_\_\_\_\_\_\_\_\_\_\_\_\_\_\_\_\_\_\_\_\_\_\_\_\_\_\_\_\_\_
global\_average\_pooling2d\_6 ( (None, 32)                0         
\_\_\_\_\_\_\_\_\_\_\_\_\_\_\_\_\_\_\_\_\_\_\_\_\_\_\_\_\_\_\_\_\_\_\_\_\_\_\_\_\_\_\_\_\_\_\_\_\_\_\_\_\_\_\_\_\_\_\_\_\_\_\_\_\_
dense\_6 (Dense)              (None, 133)               4389      
=================================================================
Total params: 39,253
Trainable params: 39,253
Non-trainable params: 0
\_\_\_\_\_\_\_\_\_\_\_\_\_\_\_\_\_\_\_\_\_\_\_\_\_\_\_\_\_\_\_\_\_\_\_\_\_\_\_\_\_\_\_\_\_\_\_\_\_\_\_\_\_\_\_\_\_\_\_\_\_\_\_\_\_

    \end{Verbatim}

    \subsubsection{(IMPLEMENTATION) Compile the
Model}\label{implementation-compile-the-model}

    \begin{Verbatim}[commandchars=\\\{\}]
{\color{incolor}In [{\color{incolor}74}]:} \PY{c+c1}{\PYZsh{}\PYZsh{}\PYZsh{} TODO: Compile the model.}
         \PY{n}{VGG19\PYZus{}model}\PY{o}{.}\PY{n}{compile}\PY{p}{(}\PY{n}{loss}\PY{o}{=}\PY{l+s+s1}{\PYZsq{}}\PY{l+s+s1}{categorical\PYZus{}crossentropy}\PY{l+s+s1}{\PYZsq{}}\PY{p}{,} \PY{n}{optimizer}\PY{o}{=}\PY{l+s+s1}{\PYZsq{}}\PY{l+s+s1}{rmsprop}\PY{l+s+s1}{\PYZsq{}}\PY{p}{,} \PY{n}{metrics}\PY{o}{=}\PY{p}{[}\PY{l+s+s1}{\PYZsq{}}\PY{l+s+s1}{accuracy}\PY{l+s+s1}{\PYZsq{}}\PY{p}{]}\PY{p}{)}
\end{Verbatim}


    \subsubsection{(IMPLEMENTATION) Train the
Model}\label{implementation-train-the-model}

Train your model in the code cell below. Use model checkpointing to save
the model that attains the best validation loss.

You are welcome to
\href{https://blog.keras.io/building-powerful-image-classification-models-using-very-little-data.html}{augment
the training data}, but this is not a requirement.

    \begin{Verbatim}[commandchars=\\\{\}]
{\color{incolor}In [{\color{incolor}78}]:} \PY{c+c1}{\PYZsh{}\PYZsh{}\PYZsh{} TODO: Train the model.}
         
         \PY{n}{checkpointer} \PY{o}{=} \PY{n}{ModelCheckpoint}\PY{p}{(}\PY{n}{filepath}\PY{o}{=}\PY{l+s+s1}{\PYZsq{}}\PY{l+s+s1}{saved\PYZus{}models/weights.best.VGG19.hdf5}\PY{l+s+s1}{\PYZsq{}}\PY{p}{,} 
                                        \PY{n}{verbose}\PY{o}{=}\PY{l+m+mi}{1}\PY{p}{,} \PY{n}{save\PYZus{}best\PYZus{}only}\PY{o}{=}\PY{k+kc}{True}\PY{p}{)}
         
         \PY{n}{VGG19\PYZus{}model}\PY{o}{.}\PY{n}{fit}\PY{p}{(}\PY{n}{train\PYZus{}VGG19}\PY{p}{,} \PY{n}{train\PYZus{}targets}\PY{p}{,} 
                   \PY{n}{validation\PYZus{}data}\PY{o}{=}\PY{p}{(}\PY{n}{valid\PYZus{}VGG19}\PY{p}{,} \PY{n}{valid\PYZus{}targets}\PY{p}{)}\PY{p}{,}
                   \PY{n}{epochs}\PY{o}{=}\PY{l+m+mi}{20}\PY{p}{,} \PY{n}{batch\PYZus{}size}\PY{o}{=}\PY{l+m+mi}{20}\PY{p}{,} \PY{n}{callbacks}\PY{o}{=}\PY{p}{[}\PY{n}{checkpointer}\PY{p}{]}\PY{p}{,} \PY{n}{verbose}\PY{o}{=}\PY{l+m+mi}{1}\PY{p}{)}
\end{Verbatim}


    \begin{Verbatim}[commandchars=\\\{\}]
Train on 6680 samples, validate on 835 samples
Epoch 1/20
6560/6680 [============================>.] - ETA: 0s - loss: 0.6476 - acc: 0.7951Epoch 00000: val\_loss improved from inf to 1.31005, saving model to saved\_models/weights.best.VGG19.hdf5
6680/6680 [==============================] - 2s - loss: 0.6487 - acc: 0.7945 - val\_loss: 1.3100 - val\_acc: 0.6635
Epoch 2/20
6660/6680 [============================>.] - ETA: 0s - loss: 0.5580 - acc: 0.8192Epoch 00001: val\_loss did not improve
6680/6680 [==============================] - 2s - loss: 0.5576 - acc: 0.8192 - val\_loss: 1.3942 - val\_acc: 0.6527
Epoch 3/20
6660/6680 [============================>.] - ETA: 0s - loss: 0.5082 - acc: 0.8375Epoch 00002: val\_loss did not improve
6680/6680 [==============================] - 2s - loss: 0.5080 - acc: 0.8374 - val\_loss: 1.4704 - val\_acc: 0.6503
Epoch 4/20
6620/6680 [============================>.] - ETA: 0s - loss: 0.4517 - acc: 0.8545Epoch 00003: val\_loss did not improve
6680/6680 [==============================] - 2s - loss: 0.4545 - acc: 0.8536 - val\_loss: 1.4241 - val\_acc: 0.6731
Epoch 5/20
6600/6680 [============================>.] - ETA: 0s - loss: 0.4057 - acc: 0.8673Epoch 00004: val\_loss did not improve
6680/6680 [==============================] - 2s - loss: 0.4078 - acc: 0.8665 - val\_loss: 1.5580 - val\_acc: 0.6635
Epoch 6/20
6540/6680 [============================>.] - ETA: 0s - loss: 0.3618 - acc: 0.8815Epoch 00005: val\_loss did not improve
6680/6680 [==============================] - 2s - loss: 0.3622 - acc: 0.8807 - val\_loss: 1.6125 - val\_acc: 0.6443
Epoch 7/20
6660/6680 [============================>.] - ETA: 0s - loss: 0.3340 - acc: 0.8901Epoch 00006: val\_loss did not improve
6680/6680 [==============================] - 2s - loss: 0.3356 - acc: 0.8891 - val\_loss: 1.6750 - val\_acc: 0.6515
Epoch 8/20
6640/6680 [============================>.] - ETA: 0s - loss: 0.2934 - acc: 0.9039Epoch 00007: val\_loss did not improve
6680/6680 [==============================] - 2s - loss: 0.2945 - acc: 0.9034 - val\_loss: 1.7196 - val\_acc: 0.6778
Epoch 9/20
6540/6680 [============================>.] - ETA: 0s - loss: 0.2656 - acc: 0.9115Epoch 00008: val\_loss did not improve
6680/6680 [==============================] - 2s - loss: 0.2660 - acc: 0.9115 - val\_loss: 1.8864 - val\_acc: 0.6383
Epoch 10/20
6580/6680 [============================>.] - ETA: 0s - loss: 0.2563 - acc: 0.9141Epoch 00009: val\_loss did not improve
6680/6680 [==============================] - 2s - loss: 0.2550 - acc: 0.9142 - val\_loss: 1.7840 - val\_acc: 0.6599
Epoch 11/20
6620/6680 [============================>.] - ETA: 0s - loss: 0.2286 - acc: 0.9224Epoch 00010: val\_loss did not improve
6680/6680 [==============================] - 2s - loss: 0.2274 - acc: 0.9228 - val\_loss: 1.8685 - val\_acc: 0.6671
Epoch 12/20
6540/6680 [============================>.] - ETA: 0s - loss: 0.2072 - acc: 0.9298Epoch 00011: val\_loss did not improve
6680/6680 [==============================] - 2s - loss: 0.2075 - acc: 0.9296 - val\_loss: 2.0457 - val\_acc: 0.6503
Epoch 13/20
6600/6680 [============================>.] - ETA: 0s - loss: 0.1909 - acc: 0.9367Epoch 00012: val\_loss did not improve
6680/6680 [==============================] - 2s - loss: 0.1905 - acc: 0.9367 - val\_loss: 1.9958 - val\_acc: 0.6790
Epoch 14/20
6660/6680 [============================>.] - ETA: 0s - loss: 0.1771 - acc: 0.9405Epoch 00013: val\_loss did not improve
6680/6680 [==============================] - 2s - loss: 0.1769 - acc: 0.9406 - val\_loss: 2.1392 - val\_acc: 0.6611
Epoch 15/20
6620/6680 [============================>.] - ETA: 0s - loss: 0.1698 - acc: 0.9444Epoch 00014: val\_loss did not improve
6680/6680 [==============================] - 2s - loss: 0.1706 - acc: 0.9443 - val\_loss: 2.1120 - val\_acc: 0.6635
Epoch 16/20
6580/6680 [============================>.] - ETA: 0s - loss: 0.1556 - acc: 0.9482Epoch 00015: val\_loss did not improve
6680/6680 [==============================] - 2s - loss: 0.1567 - acc: 0.9481 - val\_loss: 2.0951 - val\_acc: 0.6599
Epoch 17/20
6520/6680 [============================>.] - ETA: 0s - loss: 0.1561 - acc: 0.9518Epoch 00016: val\_loss did not improve
6680/6680 [==============================] - 2s - loss: 0.1540 - acc: 0.9524 - val\_loss: 2.1715 - val\_acc: 0.6862
Epoch 18/20
6660/6680 [============================>.] - ETA: 0s - loss: 0.1452 - acc: 0.9548Epoch 00017: val\_loss did not improve
6680/6680 [==============================] - 2s - loss: 0.1448 - acc: 0.9549 - val\_loss: 2.1641 - val\_acc: 0.6838
Epoch 19/20
6580/6680 [============================>.] - ETA: 0s - loss: 0.1442 - acc: 0.9527Epoch 00018: val\_loss did not improve
6680/6680 [==============================] - 2s - loss: 0.1440 - acc: 0.9525 - val\_loss: 2.3323 - val\_acc: 0.6623
Epoch 20/20
6520/6680 [============================>.] - ETA: 0s - loss: 0.1378 - acc: 0.9560Epoch 00019: val\_loss did not improve
6680/6680 [==============================] - 2s - loss: 0.1396 - acc: 0.9554 - val\_loss: 2.3239 - val\_acc: 0.6587

    \end{Verbatim}

\begin{Verbatim}[commandchars=\\\{\}]
{\color{outcolor}Out[{\color{outcolor}78}]:} <keras.callbacks.History at 0x7f25144f0b10>
\end{Verbatim}
            
    \subsubsection{(IMPLEMENTATION) Load the Model with the Best Validation
Loss}\label{implementation-load-the-model-with-the-best-validation-loss}

    \begin{Verbatim}[commandchars=\\\{\}]
{\color{incolor}In [{\color{incolor}79}]:} \PY{c+c1}{\PYZsh{}\PYZsh{}\PYZsh{} TODO: Load the model weights with the best validation loss.}
         \PY{n}{VGG19\PYZus{}model}\PY{o}{.}\PY{n}{load\PYZus{}weights}\PY{p}{(}\PY{l+s+s1}{\PYZsq{}}\PY{l+s+s1}{saved\PYZus{}models/weights.best.VGG19.hdf5}\PY{l+s+s1}{\PYZsq{}}\PY{p}{)}
         \PY{n}{VGG19\PYZus{}model}\PY{o}{.}\PY{n}{summary}\PY{p}{(}\PY{p}{)}
\end{Verbatim}


    \begin{Verbatim}[commandchars=\\\{\}]
\_\_\_\_\_\_\_\_\_\_\_\_\_\_\_\_\_\_\_\_\_\_\_\_\_\_\_\_\_\_\_\_\_\_\_\_\_\_\_\_\_\_\_\_\_\_\_\_\_\_\_\_\_\_\_\_\_\_\_\_\_\_\_\_\_
Layer (type)                 Output Shape              Param \#   
=================================================================
conv2d\_13 (Conv2D)           (None, 7, 7, 16)          32784     
\_\_\_\_\_\_\_\_\_\_\_\_\_\_\_\_\_\_\_\_\_\_\_\_\_\_\_\_\_\_\_\_\_\_\_\_\_\_\_\_\_\_\_\_\_\_\_\_\_\_\_\_\_\_\_\_\_\_\_\_\_\_\_\_\_
conv2d\_14 (Conv2D)           (None, 7, 7, 32)          2080      
\_\_\_\_\_\_\_\_\_\_\_\_\_\_\_\_\_\_\_\_\_\_\_\_\_\_\_\_\_\_\_\_\_\_\_\_\_\_\_\_\_\_\_\_\_\_\_\_\_\_\_\_\_\_\_\_\_\_\_\_\_\_\_\_\_
global\_average\_pooling2d\_6 ( (None, 32)                0         
\_\_\_\_\_\_\_\_\_\_\_\_\_\_\_\_\_\_\_\_\_\_\_\_\_\_\_\_\_\_\_\_\_\_\_\_\_\_\_\_\_\_\_\_\_\_\_\_\_\_\_\_\_\_\_\_\_\_\_\_\_\_\_\_\_
dense\_6 (Dense)              (None, 133)               4389      
=================================================================
Total params: 39,253
Trainable params: 39,253
Non-trainable params: 0
\_\_\_\_\_\_\_\_\_\_\_\_\_\_\_\_\_\_\_\_\_\_\_\_\_\_\_\_\_\_\_\_\_\_\_\_\_\_\_\_\_\_\_\_\_\_\_\_\_\_\_\_\_\_\_\_\_\_\_\_\_\_\_\_\_

    \end{Verbatim}

    \subsubsection{(IMPLEMENTATION) Test the
Model}\label{implementation-test-the-model}

Try out your model on the test dataset of dog images. Ensure that your
test accuracy is greater than 60\%.

    \begin{Verbatim}[commandchars=\\\{\}]
{\color{incolor}In [{\color{incolor}80}]:} \PY{c+c1}{\PYZsh{}\PYZsh{}\PYZsh{} TODO: Calculate classification accuracy on the test dataset.}
         
         \PY{n}{VGG19\PYZus{}predictions} \PY{o}{=} \PY{p}{[}\PY{n}{np}\PY{o}{.}\PY{n}{argmax}\PY{p}{(}\PY{n}{VGG19\PYZus{}model}\PY{o}{.}\PY{n}{predict}\PY{p}{(}\PY{n}{np}\PY{o}{.}\PY{n}{expand\PYZus{}dims}\PY{p}{(}\PY{n}{feature}\PY{p}{,} \PY{n}{axis}\PY{o}{=}\PY{l+m+mi}{0}\PY{p}{)}\PY{p}{)}\PY{p}{)} \PY{k}{for} \PY{n}{feature} \PY{o+ow}{in} \PY{n}{test\PYZus{}VGG19}\PY{p}{]}
         
         \PY{c+c1}{\PYZsh{} report test accuracy}
         \PY{n}{test\PYZus{}accuracy} \PY{o}{=} \PY{l+m+mi}{100}\PY{o}{*}\PY{n}{np}\PY{o}{.}\PY{n}{sum}\PY{p}{(}\PY{n}{np}\PY{o}{.}\PY{n}{array}\PY{p}{(}\PY{n}{VGG19\PYZus{}predictions}\PY{p}{)}\PY{o}{==}\PY{n}{np}\PY{o}{.}\PY{n}{argmax}\PY{p}{(}\PY{n}{test\PYZus{}targets}\PY{p}{,} \PY{n}{axis}\PY{o}{=}\PY{l+m+mi}{1}\PY{p}{)}\PY{p}{)}\PY{o}{/}\PY{n+nb}{len}\PY{p}{(}\PY{n}{VGG19\PYZus{}predictions}\PY{p}{)}
         \PY{n+nb}{print}\PY{p}{(}\PY{l+s+s1}{\PYZsq{}}\PY{l+s+s1}{Test accuracy: }\PY{l+s+si}{\PYZpc{}.4f}\PY{l+s+si}{\PYZpc{}\PYZpc{}}\PY{l+s+s1}{\PYZsq{}} \PY{o}{\PYZpc{}} \PY{n}{test\PYZus{}accuracy}\PY{p}{)}
\end{Verbatim}


    \begin{Verbatim}[commandchars=\\\{\}]
Test accuracy: 66.0000\%

    \end{Verbatim}

    \subsubsection{(IMPLEMENTATION) Predict Dog Breed with the
Model}\label{implementation-predict-dog-breed-with-the-model}

Write a function that takes an image path as input and returns the dog
breed (\texttt{Affenpinscher}, \texttt{Afghan\_hound}, etc) that is
predicted by your model.

Similar to the analogous function in Step 5, your function should have
three steps: 1. Extract the bottleneck features corresponding to the
chosen CNN model. 2. Supply the bottleneck features as input to the
model to return the predicted vector. Note that the argmax of this
prediction vector gives the index of the predicted dog breed. 3. Use the
\texttt{dog\_names} array defined in Step 0 of this notebook to return
the corresponding breed.

The functions to extract the bottleneck features can be found in
\texttt{extract\_bottleneck\_features.py}, and they have been imported
in an earlier code cell. To obtain the bottleneck features corresponding
to your chosen CNN architecture, you need to use the function

\begin{verbatim}
extract_{network}
\end{verbatim}

where \texttt{\{network\}}, in the above filename, should be one of
\texttt{VGG19}, \texttt{Resnet50}, \texttt{InceptionV3}, or
\texttt{Xception}.

    \begin{Verbatim}[commandchars=\\\{\}]
{\color{incolor}In [{\color{incolor}81}]:} \PY{c+c1}{\PYZsh{}\PYZsh{}\PYZsh{} TODO: Write a function that takes a path to an image as input}
         \PY{c+c1}{\PYZsh{}\PYZsh{}\PYZsh{} and returns the dog breed that is predicted by the model.}
         
         \PY{k}{def} \PY{n+nf}{VGG19\PYZus{}predict\PYZus{}breed}\PY{p}{(}\PY{n}{img\PYZus{}path}\PY{p}{)}\PY{p}{:}
             \PY{c+c1}{\PYZsh{} extract bottleneck features}
             \PY{n}{bottleneck\PYZus{}feature} \PY{o}{=} \PY{n}{extract\PYZus{}VGG19}\PY{p}{(}\PY{n}{path\PYZus{}to\PYZus{}tensor}\PY{p}{(}\PY{n}{img\PYZus{}path}\PY{p}{)}\PY{p}{)}
             \PY{c+c1}{\PYZsh{} obtain predicted vector}
             \PY{n}{predicted\PYZus{}vector} \PY{o}{=} \PY{n}{VGG19\PYZus{}model}\PY{o}{.}\PY{n}{predict}\PY{p}{(}\PY{n}{bottleneck\PYZus{}feature}\PY{p}{)}
             \PY{c+c1}{\PYZsh{} return dog breed that is predicted by the model}
             \PY{k}{return} \PY{n}{dog\PYZus{}names}\PY{p}{[}\PY{n}{np}\PY{o}{.}\PY{n}{argmax}\PY{p}{(}\PY{n}{predicted\PYZus{}vector}\PY{p}{)}\PY{p}{]}
\end{Verbatim}


    \begin{center}\rule{0.5\linewidth}{\linethickness}\end{center}

 \#\# Step 6: Write your Algorithm

Write an algorithm that accepts a file path to an image and first
determines whether the image contains a human, dog, or neither. Then, -
if a \textbf{dog} is detected in the image, return the predicted breed.
- if a \textbf{human} is detected in the image, return the resembling
dog breed. - if \textbf{neither} is detected in the image, provide
output that indicates an error.

You are welcome to write your own functions for detecting humans and
dogs in images, but feel free to use the \texttt{face\_detector} and
\texttt{dog\_detector} functions developed above. You are
\textbf{required} to use your CNN from Step 5 to predict dog breed.

Some sample output for our algorithm is provided below, but feel free to
design your own user experience!

\begin{figure}
\centering
\includegraphics{images/sample_human_output.png}
\caption{Sample Human Output}
\end{figure}

\subsubsection{(IMPLEMENTATION) Write your
Algorithm}\label{implementation-write-your-algorithm}

    \begin{Verbatim}[commandchars=\\\{\}]
{\color{incolor}In [{\color{incolor}82}]:} \PY{c+c1}{\PYZsh{}\PYZsh{}\PYZsh{} TODO: Write your algorithm.}
         \PY{c+c1}{\PYZsh{}\PYZsh{}\PYZsh{} Feel free to use as many code cells as needed.}
         
         
         \PY{k}{def} \PY{n+nf}{myAlg}\PY{p}{(}\PY{n}{image\PYZus{}path}\PY{p}{)}\PY{p}{:}
             \PY{k}{if}\PY{p}{(}\PY{n}{face\PYZus{}detector}\PY{p}{(}\PY{n}{image\PYZus{}path}\PY{p}{)}\PY{p}{)}\PY{p}{:}
                 \PY{n+nb}{print}\PY{p}{(}\PY{l+s+s2}{\PYZdq{}}\PY{l+s+s2}{Hello Human!}\PY{l+s+s2}{\PYZdq{}}\PY{p}{)}
                 \PY{c+c1}{\PYZsh{}predicted\PYZus{}dog = VGG19\PYZus{}model.predict(path\PYZus{}to\PYZus{}tensor(image\PYZus{}path))}
                 \PY{n}{predicted\PYZus{}dog} \PY{o}{=} \PY{n}{VGG19\PYZus{}predict\PYZus{}breed}\PY{p}{(}\PY{n}{image\PYZus{}path}\PY{p}{)}
                 \PY{c+c1}{\PYZsh{}return dog\PYZus{}names[np.argmax(predicted\PYZus{}dog)]}
                 \PY{k}{return} \PY{n}{predicted\PYZus{}dog}
             \PY{k}{if}\PY{p}{(}\PY{n}{dog\PYZus{}detector}\PY{p}{(}\PY{n}{image\PYZus{}path}\PY{p}{)}\PY{p}{)}\PY{p}{:}
                 \PY{n+nb}{print}\PY{p}{(}\PY{l+s+s2}{\PYZdq{}}\PY{l+s+s2}{Hello Dog!}\PY{l+s+s2}{\PYZdq{}}\PY{p}{)}
                 \PY{n}{predicted\PYZus{}dog} \PY{o}{=} \PY{n}{VGG19\PYZus{}predict\PYZus{}breed}\PY{p}{(}\PY{n}{image\PYZus{}path}\PY{p}{)}
                 \PY{k}{return} \PY{n}{predicted\PYZus{}dog}
             \PY{k}{raise} \PY{n+ne}{ValueError}\PY{p}{(}\PY{l+s+s1}{\PYZsq{}}\PY{l+s+s1}{This is neither a human nor a dog.}\PY{l+s+s1}{\PYZsq{}}\PY{p}{)}
\end{Verbatim}


    \begin{center}\rule{0.5\linewidth}{\linethickness}\end{center}

 \#\# Step 7: Test Your Algorithm

In this section, you will take your new algorithm for a spin! What kind
of dog does the algorithm think that \textbf{you} look like? If you have
a dog, does it predict your dog's breed accurately? If you have a cat,
does it mistakenly think that your cat is a dog?

\subsubsection{(IMPLEMENTATION) Test Your Algorithm on Sample
Images!}\label{implementation-test-your-algorithm-on-sample-images}

Test your algorithm at least six images on your computer. Feel free to
use any images you like. Use at least two human and two dog images.

\textbf{Question 6:} Is the output better than you expected :) ? Or
worse :( ? Provide at least three possible points of improvement for
your algorithm.

\textbf{Answer:} The output is worse than I expected to achieve. I
believe the issues involves my architecture which can be improved to
have an accuracy higher than 65\%, ideally 85\%. My architecture can be
improved by incorporating dropout layers and a learning rate which would
defiantly increase the accuracy of my architecture. In addition, the
general structure of my architecture could be improved and be trained
over more epochs.

If you want to improve the accuracy, you should perform hyper-parameter
tuning like increasing the number of epochs, trying out learning rates,
adding dropout layers, etc.

    \begin{Verbatim}[commandchars=\\\{\}]
{\color{incolor}In [{\color{incolor}83}]:} \PY{c+c1}{\PYZsh{}\PYZsh{} TODO: Execute your algorithm from Step 6 on}
         \PY{c+c1}{\PYZsh{}\PYZsh{} at least 6 images on your computer.}
         \PY{c+c1}{\PYZsh{}\PYZsh{} Feel free to use as many code cells as needed.}
         \PY{n}{img\PYZus{}path} \PY{o}{=} \PY{l+s+s2}{\PYZdq{}}\PY{l+s+s2}{images/UNADJUSTEDNONRAW\PYZus{}thumb\PYZus{}123.jpg}\PY{l+s+s2}{\PYZdq{}}
         \PY{n}{myAlg}\PY{p}{(}\PY{n}{img\PYZus{}path}\PY{p}{)}
         \PY{n}{plt}\PY{o}{.}\PY{n}{imshow}\PY{p}{(}\PY{n}{cv2}\PY{o}{.}\PY{n}{imread}\PY{p}{(}\PY{n}{img\PYZus{}path}\PY{p}{)}\PY{p}{)}
\end{Verbatim}


    \begin{Verbatim}[commandchars=\\\{\}]
Hello Human!

    \end{Verbatim}

\begin{Verbatim}[commandchars=\\\{\}]
{\color{outcolor}Out[{\color{outcolor}83}]:} <matplotlib.image.AxesImage at 0x7f251dd49ad0>
\end{Verbatim}
            
    \begin{center}
    \adjustimage{max size={0.9\linewidth}{0.9\paperheight}}{output_65_2.png}
    \end{center}
    { \hspace*{\fill} \\}
    
    \begin{Verbatim}[commandchars=\\\{\}]
{\color{incolor}In [{\color{incolor}61}]:} \PY{n}{img\PYZus{}path} \PY{o}{=} \PY{l+s+s2}{\PYZdq{}}\PY{l+s+s2}{images/UNADJUSTEDNONRAW\PYZus{}thumb\PYZus{}167.jpg}\PY{l+s+s2}{\PYZdq{}}
         \PY{n}{myAlg}\PY{p}{(}\PY{n}{img\PYZus{}path}\PY{p}{)}
         \PY{n}{plt}\PY{o}{.}\PY{n}{imshow}\PY{p}{(}\PY{n}{cv2}\PY{o}{.}\PY{n}{imread}\PY{p}{(}\PY{n}{img\PYZus{}path}\PY{p}{)}\PY{p}{)}
\end{Verbatim}


    \begin{Verbatim}[commandchars=\\\{\}]
Hello Dog!

    \end{Verbatim}

\begin{Verbatim}[commandchars=\\\{\}]
{\color{outcolor}Out[{\color{outcolor}61}]:} <matplotlib.image.AxesImage at 0x7f2516239310>
\end{Verbatim}
            
    \begin{center}
    \adjustimage{max size={0.9\linewidth}{0.9\paperheight}}{output_66_2.png}
    \end{center}
    { \hspace*{\fill} \\}
    
    \begin{Verbatim}[commandchars=\\\{\}]
{\color{incolor}In [{\color{incolor}65}]:} \PY{n}{img\PYZus{}path} \PY{o}{=} \PY{l+s+s2}{\PYZdq{}}\PY{l+s+s2}{images/UNADJUSTEDNONRAW\PYZus{}thumb\PYZus{}32.jpg}\PY{l+s+s2}{\PYZdq{}}
         \PY{n}{plt}\PY{o}{.}\PY{n}{imshow}\PY{p}{(}\PY{n}{cv2}\PY{o}{.}\PY{n}{imread}\PY{p}{(}\PY{n}{img\PYZus{}path}\PY{p}{)}\PY{p}{)}
         \PY{n}{myAlg}\PY{p}{(}\PY{n}{img\PYZus{}path}\PY{p}{)}
\end{Verbatim}


    \begin{Verbatim}[commandchars=\\\{\}]
Hello Human!

    \end{Verbatim}

\begin{Verbatim}[commandchars=\\\{\}]
{\color{outcolor}Out[{\color{outcolor}65}]:} 'Lowchen'
\end{Verbatim}
            
    \begin{center}
    \adjustimage{max size={0.9\linewidth}{0.9\paperheight}}{output_67_2.png}
    \end{center}
    { \hspace*{\fill} \\}
    
    \begin{Verbatim}[commandchars=\\\{\}]
{\color{incolor}In [{\color{incolor}64}]:} \PY{n}{img\PYZus{}path} \PY{o}{=} \PY{l+s+s2}{\PYZdq{}}\PY{l+s+s2}{images/UNADJUSTEDNONRAW\PYZus{}thumb\PYZus{}217.jpg}\PY{l+s+s2}{\PYZdq{}}
         \PY{n}{plt}\PY{o}{.}\PY{n}{imshow}\PY{p}{(}\PY{n}{cv2}\PY{o}{.}\PY{n}{imread}\PY{p}{(}\PY{n}{img\PYZus{}path}\PY{p}{)}\PY{p}{)}
         \PY{n}{myAlg}\PY{p}{(}\PY{n}{img\PYZus{}path}\PY{p}{)}
\end{Verbatim}


    \begin{Verbatim}[commandchars=\\\{\}]
Hello Dog!

    \end{Verbatim}

\begin{Verbatim}[commandchars=\\\{\}]
{\color{outcolor}Out[{\color{outcolor}64}]:} 'Lowchen'
\end{Verbatim}
            
    \begin{center}
    \adjustimage{max size={0.9\linewidth}{0.9\paperheight}}{output_68_2.png}
    \end{center}
    { \hspace*{\fill} \\}
    
    \begin{Verbatim}[commandchars=\\\{\}]
{\color{incolor}In [{\color{incolor}66}]:} \PY{n}{img\PYZus{}path} \PY{o}{=} \PY{l+s+s2}{\PYZdq{}}\PY{l+s+s2}{images/UNADJUSTEDNONRAW\PYZus{}thumb\PYZus{}26.jpg}\PY{l+s+s2}{\PYZdq{}}
         \PY{n}{plt}\PY{o}{.}\PY{n}{imshow}\PY{p}{(}\PY{n}{cv2}\PY{o}{.}\PY{n}{imread}\PY{p}{(}\PY{n}{img\PYZus{}path}\PY{p}{)}\PY{p}{)}
         \PY{n}{myAlg}\PY{p}{(}\PY{n}{img\PYZus{}path}\PY{p}{)}
\end{Verbatim}


    \begin{Verbatim}[commandchars=\\\{\}]
Hello Human!

    \end{Verbatim}

\begin{Verbatim}[commandchars=\\\{\}]
{\color{outcolor}Out[{\color{outcolor}66}]:} 'English\_cocker\_spaniel'
\end{Verbatim}
            
    \begin{center}
    \adjustimage{max size={0.9\linewidth}{0.9\paperheight}}{output_69_2.png}
    \end{center}
    { \hspace*{\fill} \\}
    
    \begin{Verbatim}[commandchars=\\\{\}]
{\color{incolor}In [{\color{incolor}67}]:} \PY{n}{img\PYZus{}path} \PY{o}{=} \PY{l+s+s2}{\PYZdq{}}\PY{l+s+s2}{images/OEVcyoa.jpg}\PY{l+s+s2}{\PYZdq{}}
         \PY{n}{plt}\PY{o}{.}\PY{n}{imshow}\PY{p}{(}\PY{n}{cv2}\PY{o}{.}\PY{n}{imread}\PY{p}{(}\PY{n}{img\PYZus{}path}\PY{p}{)}\PY{p}{)}
         \PY{n}{myAlg}\PY{p}{(}\PY{n}{img\PYZus{}path}\PY{p}{)}
\end{Verbatim}


    \begin{Verbatim}[commandchars=\\\{\}]
Hello Dog!

    \end{Verbatim}

\begin{Verbatim}[commandchars=\\\{\}]
{\color{outcolor}Out[{\color{outcolor}67}]:} 'Lowchen'
\end{Verbatim}
            
    \begin{center}
    \adjustimage{max size={0.9\linewidth}{0.9\paperheight}}{output_70_2.png}
    \end{center}
    { \hspace*{\fill} \\}
    
    \begin{Verbatim}[commandchars=\\\{\}]
{\color{incolor}In [{\color{incolor}69}]:} \PY{n}{img\PYZus{}path} \PY{o}{=} \PY{l+s+s2}{\PYZdq{}}\PY{l+s+s2}{images/German shepherd images.jpg}\PY{l+s+s2}{\PYZdq{}}
         \PY{n}{plt}\PY{o}{.}\PY{n}{imshow}\PY{p}{(}\PY{n}{cv2}\PY{o}{.}\PY{n}{imread}\PY{p}{(}\PY{n}{img\PYZus{}path}\PY{p}{)}\PY{p}{)}
         \PY{n}{myAlg}\PY{p}{(}\PY{n}{img\PYZus{}path}\PY{p}{)}
\end{Verbatim}


    \begin{Verbatim}[commandchars=\\\{\}]
Hello Human!

    \end{Verbatim}

\begin{Verbatim}[commandchars=\\\{\}]
{\color{outcolor}Out[{\color{outcolor}69}]:} 'Lowchen'
\end{Verbatim}
            
    \begin{center}
    \adjustimage{max size={0.9\linewidth}{0.9\paperheight}}{output_71_2.png}
    \end{center}
    { \hspace*{\fill} \\}
    
    \begin{Verbatim}[commandchars=\\\{\}]
{\color{incolor}In [{\color{incolor}70}]:} \PY{n}{img\PYZus{}path} \PY{o}{=} \PY{l+s+s2}{\PYZdq{}}\PY{l+s+s2}{images/Labrador\PYZus{}retriever\PYZus{}06449.jpg}\PY{l+s+s2}{\PYZdq{}}
         \PY{n}{plt}\PY{o}{.}\PY{n}{imshow}\PY{p}{(}\PY{n}{cv2}\PY{o}{.}\PY{n}{imread}\PY{p}{(}\PY{n}{img\PYZus{}path}\PY{p}{)}\PY{p}{)}
         \PY{n}{myAlg}\PY{p}{(}\PY{n}{img\PYZus{}path}\PY{p}{)}
\end{Verbatim}


    \begin{Verbatim}[commandchars=\\\{\}]
Hello Dog!

    \end{Verbatim}

\begin{Verbatim}[commandchars=\\\{\}]
{\color{outcolor}Out[{\color{outcolor}70}]:} 'Lowchen'
\end{Verbatim}
            
    \begin{center}
    \adjustimage{max size={0.9\linewidth}{0.9\paperheight}}{output_72_2.png}
    \end{center}
    { \hspace*{\fill} \\}
    

    % Add a bibliography block to the postdoc
    
    
    
    \end{document}
